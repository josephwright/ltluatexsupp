

\edef\etatcatcode{\the\catcode`\@}
\catcode`\@=11

\ifx\e@alloc\@undefined\else
\expandafter\endinput
\fi

%
% fixes to etex.src, 
% These couldand probably should be made directly in an
% update to etex.src which already has some luatex-specific
% code, but does not define the correct range for luatex.

% 2015-07-13 higher range in luatex
\edef \et@xmaxregs {\ifx\directlua\@undefined 32768\else 65536\fi}

\count 270=\et@xmaxregs % locally allocates \count registers 32767, 32766, ...
\count 271=\et@xmaxregs % ditto for \dimen registers
\count 272=\et@xmaxregs % ditto for \skip registers
\count 273=\et@xmaxregs % ditto for \muskip registers
\count 274=\et@xmaxregs % ditto for \box registers
\count 275=\et@xmaxregs % ditto for \toks registers
\count 276=\et@xmaxregs % ditto for \marks classes

%%%%%%%%%%%%%%%%%%%%%%%%%%%%%%%%%%%%%%%%%%%%%%%%%%%%%%




% define\e@alloc as in latex (the existing macros in etex.src
% hard to extend to further register types as they assume specific
% 26x and 27x count range. For compatibility the existing register
% allocation is not changed.


\def\e@alloc#1#2#3#4#5#6{%
  \global\advance#3\@ne
  \e@ch@ck{#3}{#4}{#5}#1%
  \allocationnumber#3\relax
  \global#2#6\allocationnumber
  \wlog{\string#6=\string#1\the\allocationnumber}}%
\gdef\e@ch@ck#1#2#3#4{%
  \ifnum#1<#2\else
    \ifnum#1=#2\relax
      #1\@cclvi
      \ifx\count#4\advance#1 10 \fi
    \fi
    \ifnum#1<#3\relax
    \else
      \errmessage{No room for a new #4}%
    \fi
  \fi}%


%%%%%%
\long\def \@gobble #1{}
\long\def\@firstofone#1{#1}
\def\makeatother{\catcode`\@12\relax}
\input ltluatex.sty


% fix up allocations not to clash with etex.src
% \count registers 256-259 and 267-269 are not (yet) used
% 

% ltluatex uses 258 and 259
% also uses 260 and 261 but change them to 267 and 268
\def\newluafunction{%
  \e@alloc\luafunction\e@alloc@chardef
    {\count267}\m@ne\e@alloc@top
}
\count267=\z@
\def\newwhatsit#1{%
  \e@alloc\whatsit\e@alloc@chardef
    {\count268}\m@ne\e@alloc@top#1%
}
\count268=\z@

\directlua{
local function new_whatsit(name)
  tex_setcount("global", 267, tex_count[267] + 1)
  if tex_count[267] > 65534 then
    latex_error("No room for a new custom whatsit")
    return -1
  end
  texio_write_nl("Custom whatsit " .. name .. " = " .. tex_count[267])
  return tex_count[267]
end
latex.new_whatsit = new_whatsit
}

\catcode`\@=\etatcatcode\relax


% Prevent luatexbase loading when it is referenced
% by fontspec.
\expandafter\def\csname luatexbase@sty@endinput\endcsname{}

% Simple require wrapper so that luaotfload does get loaded.
\def\RequireLuaModule#1{\directlua{require("#1")}}


\directlua{
% Fake a mininmal amount of the luatexbase
% table to avoid needing to edit luaotfload.
% Once more stable hopefully this will not be needed.
luatexbase={}
%
% copy Module declaration
luatexbase.provides_module = latex.provides_module
%
% luatexbase version records name in the attributes table.
function luatexbase.new_attribute (name,...)
local n = latex.new_attribute(name)
luatexbase.attributes[name]=n
return n
end
%
% New allocator does not maintain a separate table
% of attribute names, relies on registernumber
% for that lookup, attributes declared in lua
% use same sequence but calling lua code responsible
% for keeping track of names.
luatexbase.attributes={}
%
%
% ltluatex does not copy the user defined callback mechanism
% luaotfload needs (or at least uses) a simple one.
% Here provide a very simple callback mechanism based on a
% table of functions.
% No provision for having lists of functions in a callback
% But this is enough here.
luatexbase.callbacks={}
function luatexbase.create_callback(a,b,c)
luatexbase.callbacks[a]=b
end
function luatexbase.reset_callback(a)
luatexbase.callbacks[a]=nil
end
function luatexbase.call_callback(name,...)
luatexbase.callbacks[name](...)
end
function luatexbase.add_to_callback(a,b,c,d)
luatexbase.add_to_callback(a,b,c)
end
function luatexbase.add_to_callback(a,b,c)
if(luatexbase.callbacks[a] == nil)
then
  latex.add_to_callback(a,b,c)
else
  luatexbase.callbacks[a]=b
end
end
}

\input{luaotfload.sty}

\font\llrm="Linux Libertine O"
\font\lbsf="Linux Biolinum O"

{\llrm abc Difficult Volume AVAW}

{\lbsf xyz find fling Win}

\bye