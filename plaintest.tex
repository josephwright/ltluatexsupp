% !TeX program = LuaTeX

\edef\etatcatcode{\the\catcode`\@}
\catcode`\@=11

\ifx\e@alloc\@undefined\else
\expandafter\endinput
\fi

%
% fixes to etex.src, 
% These couldand probably should be made directly in an
% update to etex.src which already has some luatex-specific
% code, but does not define the correct range for luatex.

% 2015-07-13 higher range in luatex
\edef \et@xmaxregs {\ifx\directlua\@undefined 32768\else 65536\fi}

\count 270=\et@xmaxregs % locally allocates \count registers 32767, 32766, ...
\count 271=\et@xmaxregs % ditto for \dimen registers
\count 272=\et@xmaxregs % ditto for \skip registers
\count 273=\et@xmaxregs % ditto for \muskip registers
\count 274=\et@xmaxregs % ditto for \box registers
\count 275=\et@xmaxregs % ditto for \toks registers
\count 276=\et@xmaxregs % ditto for \marks classes

%%%%%%%%%%%%%%%%%%%%%%%%%%%%%%%%%%%%%%%%%%%%%%%%%%%%%%




% define\e@alloc as in latex (the existing macros in etex.src
% hard to extend to further register types as they assume specific
% 26x and 27x count range. For compatibility the existing register
% allocation is not changed.


\def\e@alloc#1#2#3#4#5#6{%
  \global\advance#3\@ne
  \e@ch@ck{#3}{#4}{#5}#1%
  \allocationnumber#3\relax
  \global#2#6\allocationnumber
  \wlog{\string#6=\string#1\the\allocationnumber}}%
\gdef\e@ch@ck#1#2#3#4{%
  \ifnum#1<#2\else
    \ifnum#1=#2\relax
      #1\@cclvi
      \ifx\count#4\advance#1 10 \fi
    \fi
    \ifnum#1<#3\relax
    \else
      \errmessage{No room for a new #4}%
    \fi
  \fi}%


%%%%%%
\long\def \@gobble #1{}
\long\def\@firstofone#1{#1}
\def\makeatother{\catcode`\@12\relax}
\input ltluatex.sty


% fix up allocations not to clash with etex.src
% \count registers 256-259 and 267-269 are not (yet) used
% 

% ltluatex uses 258 and 259
% also uses 260 and 261 but change them to 267 and 268
\def\newluafunction{%
  \e@alloc\luafunction\e@alloc@chardef
    {\count267}\m@ne\e@alloc@top
}
\count267=\z@
\def\newwhatsit#1{%
  \e@alloc\whatsit\e@alloc@chardef
    {\count268}\m@ne\e@alloc@top#1%
}
\count268=\z@

\directlua{
local function new_whatsit(name)
  tex_setcount("global", 267, tex_count[267] + 1)
  if tex_count[267] > 65534 then
    latex_error("No room for a new custom whatsit")
    return -1
  end
  texio_write_nl("Custom whatsit " .. name .. " = " .. tex_count[267])
  return tex_count[267]
end
latex.new_whatsit = new_whatsit
}

\catcode`\@=\etatcatcode\relax

% emu-luatexbase.sty v 0.01 2015/07/14
% minimal luatexbase emulation, for testing with
% current luaotfload package
%
% Prevent luatexbase loading when it is referenced
\expandafter\def\csname luatexbase@sty@endinput\endcsname{}
\expandafter\def\csname ver@luatexbase.sty\endcsname{}
%
% Simple require wrapper so that luaotfload does get loaded.
\def\RequireLuaModule#1{\directlua{require("#1")}}

\directlua{
% Fake a mininmal amount of the luatexbase
% table to avoid needing to edit luaotfload.
% Once more stable hopefully this will not be needed.
luatexbase={}
%
% copy Module declaration
luatexbase.provides_module = latex.provides_module
%
% luatexbase version records name in the attributes table.
function luatexbase.new_attribute (name,...)
local n = latex.new_attribute(name)
luatexbase.attributes[name]=n
return n
end
%
% New allocator does not maintain a separate table
% of attribute names, relies on registernumber
% for that lookup, attributes declared in lua
% use same sequence but calling lua code responsible
% for keeping track of names. Use a metatable to look up
% the names for any tex-allocated attributes.
luatexbase.attributes=setmetatable(
{},
{
__index = function(t,key)
return latex.registernumber(key) or nil
end}
)
%
%
% callbacks
luatexbase.create_callback=latex.create_callback
luatexbase.call_callback=latex.call_callback
luatexbase.add_to_callback=latex.add_to_callback
% probably this shouldn't be allowed on primitive callbacks
% but this version allows otfload to run.
function luatexbase.reset_callback(a)
latex.disable_callback(a)
end
%
% fontspec uses a latex-package catcode table
% this is called catcodetable@atletter in ltluatex.
% We also need to emulate the catcodetables table.
luatexbase.catcodetables={}
luatexbase.catcodetables['latex-package'] =
      latex.registernumber("catcodetable@atletter")
}


\input{luaotfload.sty}

\font\llrm="Linux Libertine O"
\font\lbsf="Linux Biolinum O"

{\llrm abc Difficult Volume AVAW}

{\lbsf xyz find fling Win}


\def\showcs#1{%
  \message{#1:
    @=\the\catcode`\@, 
    space=\the\catcode32,
    Z=\the\catcode`\Z}%
}

\newcatcodetable\ctbl
\newattribute\lattr

\directlua{registernumber = luatexbase.registernumber}

\showcs{A1}

\pushcatcodes

\catcode`\Z=13

\showcs{B1}

\pushcatcodes

\catcodetable\CatcodeTableExpl

\showcs{C1}

\catcode`\Z=4

\showcs{C2}


\popcatcodes

\showcs{B3}

\popcatcodes

\showcs{A4}

\directlua{print("\string\nctbl" .. " is register:  " 
.. (registernumber("ctbl") or "??"))}
\directlua{print("\string\nlattr" .. " is register:"
.. (registernumber("lattr") or "??"))}

\newbox\mybox
{
\setbox2\hbox{xyz}
\lattr=5
\setbox0\hbox{abc}
\lattr=6
\setbox\mybox\hbox{123}

\directlua{
local n0=tex.getbox(0)
local n2=tex.getbox(2)
local nmybox=tex.getbox(registernumber("mybox"))
local a = registernumber("lattr")
print("\string\n test lattr on box 0: " 
.. (node.has_attribute(n0,a) or "unset"))
print("\string\n test lattr on box 2: "
 .. (node.has_attribute(n2,a) or "unset"))
print("\string\n test lattr on box mybox: " 
.. (node.has_attribute(nmybox,a) or "unset"))
}
}

\directlua{
function zzzX (s)
return string.gsub(s,"X", "[Capital x was here]")
end
function zzzC (s)
return string.gsub(s,"C", "[Capital c was here]")
end
luatexbase.add_to_callback('process_input_buffer',zzzX,"testing X")
luatexbase.add_to_callback('process_input_buffer',zzzC,"testing C")
}

\message{123XYZ456ABC}


\newwhatsit\fromtexwhatsita
\message{fromtexwhatsita=\the\fromtexwhatsita}

\directlua{
luatexbase.new_whatsit("fromluawhatsita")
}

\newwhatsit\fromtexwhatsitb

\message{fromtexwhatsitb=\the\fromtexwhatsitb}

\directlua{
luatexbase.new_whatsit("fromluawhatsitb")
}

\newcount\lcount
\lcount=88
\directlua{print("\string\nlcount" .. " is register:  " 
.. registernumber("lcount") )}

\bye