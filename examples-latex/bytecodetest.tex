
\edef\etatcatcode{\the\catcode`\@}
\catcode`\@=11

\ifx\e@alloc\@undefined\else
\expandafter\endinput
\fi

%
% fixes to etex.src, 
% These couldand probably should be made directly in an
% update to etex.src which already has some luatex-specific
% code, but does not define the correct range for luatex.

% 2015-07-13 higher range in luatex
\edef \et@xmaxregs {\ifx\directlua\@undefined 32768\else 65536\fi}

\count 270=\et@xmaxregs % locally allocates \count registers 32767, 32766, ...
\count 271=\et@xmaxregs % ditto for \dimen registers
\count 272=\et@xmaxregs % ditto for \skip registers
\count 273=\et@xmaxregs % ditto for \muskip registers
\count 274=\et@xmaxregs % ditto for \box registers
\count 275=\et@xmaxregs % ditto for \toks registers
\count 276=\et@xmaxregs % ditto for \marks classes

%%%%%%%%%%%%%%%%%%%%%%%%%%%%%%%%%%%%%%%%%%%%%%%%%%%%%%




% define\e@alloc as in latex (the existing macros in etex.src
% hard to extend to further register types as they assume specific
% 26x and 27x count range. For compatibility the existing register
% allocation is not changed.


\def\e@alloc#1#2#3#4#5#6{%
  \global\advance#3\@ne
  \e@ch@ck{#3}{#4}{#5}#1%
  \allocationnumber#3\relax
  \global#2#6\allocationnumber
  \wlog{\string#6=\string#1\the\allocationnumber}}%
\gdef\e@ch@ck#1#2#3#4{%
  \ifnum#1<#2\else
    \ifnum#1=#2\relax
      #1\@cclvi
      \ifx\count#4\advance#1 10 \fi
    \fi
    \ifnum#1<#3\relax
    \else
      \errmessage{No room for a new #4}%
    \fi
  \fi}%


%%%%%%
\long\def \@gobble #1{}
\long\def\@firstofone#1{#1}
\def\makeatother{\catcode`\@12\relax}
\input ltluatex.sty


% fix up allocations not to clash with etex.src
% \count registers 256-259 and 267-269 are not (yet) used
% 

% ltluatex uses 258 and 259
% also uses 260 and 261 but change them to 267 and 268
\def\newluafunction{%
  \e@alloc\luafunction\e@alloc@chardef
    {\count267}\m@ne\e@alloc@top
}
\count267=\z@
\def\newwhatsit#1{%
  \e@alloc\whatsit\e@alloc@chardef
    {\count268}\m@ne\e@alloc@top#1%
}
\count268=\z@

\directlua{
local function new_whatsit(name)
  tex_setcount("global", 267, tex_count[267] + 1)
  if tex_count[267] > 65534 then
    latex_error("No room for a new custom whatsit")
    return -1
  end
  texio_write_nl("Custom whatsit " .. name .. " = " .. tex_count[267])
  return tex_count[267]
end
latex.new_whatsit = new_whatsit
}

\catcode`\@=\etatcatcode\relax


% declare a chunk name
% the chunk number is (curently?) just stored in lua, so pass it to TeX
\directlua{
c1=luatexbase.new_chunkname('chunk one')
tex.sprint("\string\\chardef\string\\mychnk=" .. c1 .."\string\\relax")
}


% trace shows "chunk one"
\directlua\mychnk{
function foo ()
  print(debug.traceback())
end
foo()
}

% trace shows "chunk one" and "new chunk"
\directlua name{new chunk}{foo()}


\directlua{
b1=luatexbase.new_bytecode()
}

\directlua{
b2=luatexbase.new_bytecode("ss")
}


\directlua{
function myf (n,c)
 texio.write_nl("myf called with argument:" .. n)
 c = c + 2
 texio.write_nl("myf c is now:" .. c)
 texio.write_nl(debug.traceback())
end
}

% call directly
\directlua{myf('x',2)}

% byte compile
\directlua{
lua.bytecode[b1]=myf
}

% call byte compiled version
\directlua\mychnk{lua.bytecode[b1]('y',5)}


% testing allocating chunk names from TeX and Lua


\newluachunk\mych
\newluachunk\mychb
\directlua{
luatexbase.new_chunkname("luachunk")
}
\directlua \mych {
function test_stack ()
print(debug.traceback())
end
}
\directlua \mychb{
test_stack()
}
\directlua name{luachunk}{
test_stack()
}

\stop
