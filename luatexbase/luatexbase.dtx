% \iffalse meta-comment
%
% Copyright 2015
% The LaTeX3 Project and any individual authors listed elsewhere
% in this file.
%
% It may be distributed and/or modified under the conditions of
% the LaTeX Project Public License (LPPL), either version 1.3c of
% this license or (at your option) any later version.  The latest
% version of this license is in the file:
%
%   http://www.latex-project.org/lppl.txt
%
%
%
%<emu>\ifx\BeginCatcodeRegime\undefined\else\expandafter\endinput\fi
%<plain>\ifx\newluafunction\undefined\else\expandafter\endinput\fi
%<tex,sty>\ifx
%<sty>  \ProvidesPackage\undefined\begingroup\def\ProvidesPackage
%<tex>  \ProvidesFile\undefined\begingroup\def\ProvidesFile
%<tex,sty>  #1#2[#3]{\endgroup\immediate\write16{Package: #1 #3}}
%<tex,sty>\fi
%<plain>\ProvidesFile{ltluatex.tex}
%<emu>\ProvidesPackage{luatexbase}
%<emu-cmp>\ProvidesPackage{luatexbase-compat}
%<emu-mod>\ProvidesPackage{luatexbase-modutils}
%<emu-loa>\ProvidesPackage{luatexbase-loader}
%<emu-reg>\ProvidesPackage{luatexbase-regs}
%<emu-att>\ProvidesPackage{luatexbase-attr}
%<emu-cct>\ProvidesPackage{luatexbase-cctb}
%<emu-mcb>\ProvidesPackage{luatexbase-mcb}
%<*driver>
\ProvidesFile{luatexbase.dtx}
%</driver>
%<*tex,sty>
[2015/08/01 v1.0a
%</tex,sty>
%<plain>  luatex support for plain TeX (core + catcode stack)
%<emu>  luatexbase interface to luatex
%<emu-cmp>  luatexbase interface to luatex (legacy subpackage)
%<emu-mod>  luatexbase interface to luatex (legacy subpackage)
%<emu-loa>  luatexbase interface to luatex (legacy subpackage)
%<emu-reg>  luatexbase interface to luatex (legacy subpackage)
%<emu-cct>  luatexbase interface to luatex (legacy subpackage)
%<emu-mcb>  luatexbase interface to luatex (legacy subpackage)
%<*tex,sty>
]
%</tex,sty>
%<*driver>
\documentclass{ltxdoc}
\GetFileInfo{luatexbase.dtx}
\begin{document}
\title{\filename\\(Lua\TeX{}-specific support, luatexbase interface)}
\author{David Carlisle and Joseph Wright\footnote{Significant portions
  of the code here are adapted/simplified from the packages \textsf{luatex} and
  \textsf{luatexbase} written by Heiko Oberdiek, \'{E}lie Roux,
  Manuel P\'{e}gouri\'{e}-Gonnar and Philipp Gesang.}}
\date{\filedate}
\maketitle
\setcounter{tocdepth}{2}
\tableofcontents
\DocInput{\filename}
\end{document}
%</driver>
% \fi
%
% \CheckSum{0}
%
% \section{Overview}
%
% Lua\TeX{} adds a number of engine-specific functions to \TeX{}. Several of
% these require set up that is best done in the kernel or need related support
% functions. This file provides \emph{basic} support for Lua\TeX{} at the
% \LaTeXe{} kernel level plus as a loadable package which can be used with plain
% \TeX{} and \LaTeX{}. For \LaTeX{}, the package adds a small number of commands
% which do not fit into the scope of the kernel.
%
% This file contains code for both \TeX{} (to be stored as part of the format)
% and Lua (to be loaded at the start of each job). In the Lua code, the kernel
% uses the namespace |latex| which is \emph{reserved for use by the team}.
%
% The following |\count| registers are used here for register allocation:
% \begin{itemize}
%  \item[\texttt{\string\e@alloc@attribute@count}] Attributes (default 258)
%  \item[\texttt{\string\e@alloc@ccodetable@count}] Category code tables (default 259)
%  \item[\texttt{\string\e@alloc@luafunction@count}] Lua functions (default 260)
%  \item[\texttt{\string\e@alloc@whatsit@count}] User whatsits (default 261)
% \end{itemize}
% (|\count256| is used for |\newmarks| allocation and |\count257|
% is used for\break
% |\newXeTeXintercharclass| with Xe\TeX{}, with code defined in
% \texttt{ltfinal.dtx}).
% With any \LaTeXe{} kernel from 2015 onward these registers are part of
% the block in the extended area reserved by the kernel (prior to 2015 the
% \LaTeXe{} kernel did not provide any functionality for the extended
% allocation area).
%
% \section{Core \TeX{} functionality}
%
% The commands defined here are defined in |ltluatex.def| for
% possible inclusion in a future \LaTeX\ format, however for ease of
% testing they are also currently defined in the package file
% |ltluatex.sty| along with the additional commands described in the
% next section.
%
% \DescribeMacro{\newattribute}
% |\newattribute{|\meta{attribute}|}|\\
% Defines a named \cs{attribute}, indexed from~$1$
% (\emph{i.e.}~|\attribute0| is never defined). Attributes initially
% have the marker value |-"7FFFFFFF| (`unset') set by the engine.
%
% \noindent
% \DescribeMacro{\newcatcodetable}
% |\newcatcodetable{|\meta{catcodetable}|}|\\
% Defines a named \cs{catcodetable}, indexed from~$1$
% (|\catcodetable0| is never assigned). A new catcode table will be
% populated with exactly those values assigned by Ini\TeX{} (as described
% in the Lua\TeX{} manual).
%
% \noindent
% \DescribeMacro{\newluafunction}
% |\newluafunction{|\meta{function}|}|\\
% Defines a named \cs{luafunction}, indexed from~$1$ (Lua indexes from
% $1$ so |\luafunction0| is not available).
%
% \noindent
% \DescribeMacro{\newwhatsit}
% |\newwhatsit{|\meta{whatsit}|}|\\
% Defines a custom \cs{whatsit}, indexed from~$1$.
%
% \noindent
% \DescribeMacro{\catcodetable@initex}
% \DescribeMacro{\catcodetable@string}
% \DescribeMacro{\catcodetable@latex}
% \DescribeMacro{\catcodetable@atletter}
% Predefined category code tables with the obvious assignments. Note
% that the |latex| and |atletter| tables set the full Unicode range
% to the codes predefined by the kernel.
%
% \section{The \texttt{ltluatex} Package \TeX{} functionality}
%
% \noindent
% \DescribeMacro{\setattribute}
% \DescribeMacro{\unsetattribute}
% |\setattribute{|\meta{attribute}|}{|\meta{value}|}|\\
% |\unsetattribute{|\meta{attribute}|}|\\
% Set and unset attributes in a manner analogous to |\setlength|. Note that
% attributes take a marker value when unset so this operation is distinct
% from setting the value to zero.
%
% \noindent
% \DescribeMacro{\setcatcodetable}
% |\setcatcodetable{|\meta{table}|}{|\meta{catcodes}|}|\\
% Sets the \meta{table} (which must have been previously defined) to
% apply the \meta{catcodes} specified.
%
% \noindent
% \DescribeMacro{\setrangecatcode}
% |\setrangecatcode{|\meta{start}|}{|\meta{end}|}{|\meta{catcode}|}|\\
% Sets all characters in the range \meta{start}--\meta{end} inclusive to
% have the \meta{catcode} specified.
%
% \noindent
% \DescribeMacro{\@pushcatcodes}
% \DescribeMacro{\@popcatcodes}
% |\@pushcatcodetable|\\
% |\@popcatcodetable|\\
% This pair of commands enable the current category code r\'{e}gime to
% be saved and restored meaning that arbitrary catcode changes can be made.
% This functionality will normally be used in concert with applying
% catcode tables. For example
% \begin{verbatim}
% \catcode`\Z=4 %
% \@pushcatcodetable
% \catcodetable\catcodetable@latex
% % Code here
% \@popcatcodetable
% \showthe\catcode`\Z
% \end{verbatim}
% will ensure that the `content' is set with normal category codes but
% allow restoration of the non-standard codes at the conclusion. Importantly,
% it does not require that anything is known about the catcode situation in
% advance (\emph{cf.}~a more traditional approach to saving the state of
% targeting characters).
%
% \section{The \texttt{luatexbase} package interface}
%
% The interfaces defined here are closely modelled on the earlier
% \textsf{luatexbase} package, which in turn copies parts of the
% \textsf{luatex} package.
% A closer emulation of the existing \textsf{luatexbase} is provided by the
% package currently called |emu-luatexbase|. Having a different name
% allows both packages to be easily used while testing, although
% possibly |emu-luatexbase| can be the source of an updated
% \textsf{luatexbase} package, depending on how things turn out.
%
% \subsection{\TeX}
% \subsubsection{Alternative catcode table handling}
%
% \noindent
% \DescribeMacro{\CatcodeTableIniTeX}
% \DescribeMacro{\CatcodeTableString}
% \DescribeMacro{\CatcodeTableLaTeX}
% \DescribeMacro{\CatcodeTableLaTeXAtLetter}
% \DescribeMacro{\CatcodeTableOther}
% \DescribeMacro{\CatcodeTableExpl}
% Alternative names for some predefined catcodetables, and some
% additional ones are declared.
%
% \noindent
% \DescribeMacro{\BeginCatcodeRegime}
% \DescribeMacro{\EndCatcodeRegime}
% A simple wrapper around |\@pushcatcodetable| providing aslightly
% different interface.
%
% \noindent
% \DescribeMacro{\PushCatcodeTableNumStack}
% \DescribeMacro{\PopCatcodeTableNumStack}
% These are defined to be aliases for |\@pushcatcodetable| and
% |\@popcatcodetable| although the actual implemenatation is quite different
% to the older packages, the use of the commands should match.
%
% \subsection{Lua}
% \subsubsection{predefined catcode table names}
% As in the previous packages the |catcodetables| array is extended
% with aliases for the predefined tables so you can use |catcodetables.expl| in
% addition to |catcodetables.CatcodeTableExpl|.
%
% \subsubsection{callback queries}
% In the earlier interfaces the function
% |luatexbase.priority_in_callback|
% is provided to return a number indicating th eposition of a
% specified function in a callback list. However it is usually used
% just as a boolean test that the function is registered with the
% callback. |ltluatex| does not expose the priority numbers so this
% function is defined as an alias to the function
% |luatexbase.in_callback| which returns a boolean.
%
% \subsubsection{Module declaration}
% The |luatexbase| version of |provides_module| returns a list of log
% and error functions so that it is usually called as:\\
% |local err, warning, info, log = luatexbase.provides_module({name=..|
%
% \section{Plain \TeX\ interface}
%
% The |ltluatex| interface may be used with plain \TeX\ using 
% |\input ltluatex| this inputs |ltluatex.tex| which inputs |etex.src|
% if it is not already input, and then defines some internal commands to
% allow the \emph{extended} interafce defined by |ltluatex.sty| to be
% defined.
%
% The `luatexbase` package interface may also be used in plain \TeX,
% as before by simply inputting the package, currently |\input emu-luatexbase.sty|.
%
% \section{Lua functionality}
%
% \begingroup
%
% \begingroup\lccode`~=`_
% \lowercase{\endgroup\let~}_
% \catcode`_=12
%
% \section{Allocators in Lua}
%
% \DescribeMacro{luatexbase.new_attribute}
% |luatexbase.new_attribute(|\meta{attribute}|)|\\
% Returns an allocation number for the \meta{attribute}, indexed from~$1$.
% The attribute will be initialised with the marker value |-"7FFFFFFF|
% (`unset'). The attribute allocation sequence is shared with the \TeX{}
% code but this function does \emph{not} define a token using
% |\attributedef|.
% The attribute name is recorded in the |attributes| table. A
% metatable is provided so that the table syntax can be used
% consistently for attributes declared in \TeX\ or lua.
%
% \noindent
% \DescribeMacro{luatexbase.new_whatsit}
% |luatexbase.new_whatsit(|\meta{whatsit}|)|\\
% Returns an allocation number for the custom \meta{whatsit}, indexed from~$1$.
%
% \noindent
% \DescribeMacro{luatexbase.new_bytecode}
% |luatexbase.new_bytecode(|\meta{bytecode}|)|\\
% Returns an allocation number for a bytecode register, indexed from~$1$.
% The optional \meta{name} argument is just used for logging.
%
% \noindent
% \DescribeMacro{luatexbase.new_chunkname}
% |luatexbase.new_chunkname(|\meta{chunkname}|)|\\
% Returns an allocation number for a lua chunk name for use with 
% |\directlua| and |\latelua|, indexed from~$1$.
% The number is returned and also \meta{name} argument is added to the
% |lua.name| array at that index.
%
%
% \section{Lua access to \TeX{} register numbers}
%
% \DescribeMacro{luatexbase.registernumber}
% |luatexbase.registernumer(|\meta{name}|)|\\
% Sometimes (notably in the case of Lua attributes) it is necessary to
% access a register \emph{by number} that has been allocated by \TeX{}.
% This package provides a function to look up the relevant number
% using Lua\TeX{}'s internal tables. So after for example
% |\newattribute\myattrib|, |\myattrib| would be defined by (say)
% |\myattrib=\attribute15|. The function call |luatexbase.registernumer("myattrib")|
% will then return the register number, $15$ in this case. If the string passed
% as argument does not correspond to a token defined by |\attributedef|,
% |\countdef| or similar commands, the Lua value |false| is returned.
%
% As an example, consider the input:
%\begin{verbatim}
% \newcommand\test[1]{%
% \typeout{#1: \expandafter\meaning\csname#1\endcsname^^J
% \space\space\space\space
% \directlua{tex.write(luatexbase.registernumber("#1") or "bad input")}%
% }}
%
% \test{undefinedrubbish}
%
% \test{space}
%
% \test{hbox}
%
% \test{@MM}
%
% \test{@tempdima}
% \test{@tempdimb}
%
% \test{strutbox}
%
% \test{sixt@@n}
%
% \attrbutedef\myattr=12
% \myattr=200
% \test{myattr}
%
%\end{verbatim}
%
% If the demonstration code is processed with Lua\LaTeX{} then the following
% would be produced in the log and terminal output.
%\begin{verbatim}
% undefinedrubbish: \relax
%      bad input
% space: macro:->
%      bad input
% hbox: \hbox
%      bad input
% @MM: \mathchar"4E20
%      20000
% @tempdima: \dimen14
%      14
% @tempdimb: \dimen15
%      15
% strutbox: \char"B
%      11
% sixt@@n: \char"10
%      16
% myattr: \attribute12
%      12
%\end{verbatim}
%
% Notice how undefined commands, or commands unrelated to registers
% do not produce an error, just return |false| and so print
% |bad input| here. Note also that commands defined by |\newbox| work and
% return the number of the box register even though the actual command
% holding this number is a |\chardef| defined token (there is no
% |\boxdef|).
%
% \subsection{Module utilities}
%
% \DescribeMacro{luatexbase.provides_module}
% |luatexbase.provides_module(|\meta{info}|)|\\
% This function is used by modules to identify themselves; the |info| should be
% a table containing information about the module. The required field
% |name| must contain the name of the module. It is recommended to provide a
% field |date| in the usual \LaTeX{} format |yyyy/mm/dd|. Optional fields
% |version| (a string) and |description| may be used if present. This
% information will be recorded in the log. Other fields are ignored.
%
% \noindent
% \DescribeMacro{luatexbase.module_info}
% \DescribeMacro{luatexbase.module_warning}
% \DescribeMacro{luatexbase.module_error}
% |luatexbase.module_info(|\meta{module}, \meta{text}|)|\\
% |luatexbase.module_warning(|\meta{module}, \meta{text}|)|\\
% |luatexbase.module_error(|\meta{module}, \meta{text}|)|\\
% These functions are similar to \LaTeX{}'s |\PackageError|, |\PackageWarning|
% and |\PackageInfo| in the way they format the output.  No automatic line
% breaking is done, you may still use |\n| as usual for that, and the name of
% the package will be prepended to each output line.
%
% Note that |luatexbase.module_error| raises an actual Lua error with |error()|,
% which currently means a call stack will be dumped. While this may not
% look pretty, at least it provides useful information for tracking the
% error down.
%
% \subsection{Callback management}
%
% \noindent
% \DescribeMacro{luatexbase.add_to_callback}
% |luatexbase.add_to_callback(|^^A
% \meta{callback}, \meta{function}, \meta{description}|)|
% Registers the \meta{function} into the \meta{callback} with a textual
% \meta{description} of the function. Functions are inserted into the callback
% in the order loaded.
%
% \noindent
% \DescribeMacro{luatexbase.remove_from_callback}
% |luatexbase.remove_from_callback(|\meta{callback}, \meta{description}|)|
% Removes the function with \meta{description} from the \meta{callback}.
% The removed function and its description 
% are returned as the results of this function.
%
% \noindent
% \DescribeMacro{luatexbase.in_callback}
% |luatexbase.in_callback(|\meta{callback}, \meta{description}|)|
% Checks if the \meta{description} matches one of the functions added
% to the list for the \meta{callback}, returning a boolean value.
%
% \noindent
% \DescribeMacro{luatexbase.disable_callback}
% |luatexbase.disable_callback(|\meta{callback}|)|
% Sets the \meta{callback} to \texttt{false} as described in the Lua\TeX{}
% manual for the underlying \texttt{callback.register} built-in. Callbacks
% will only be set to false (and thus be skipped entirely) if there are
% no functions registered using the callback.
%
% \noindent
% \DescribeMacro{luatexbase.callback_descriptions}
% A list of the descriptions of functions registered to the specified
% callback is returned. |{}| is returned if there are no functions registered.
%
% \noindent
% \DescribeMacro{luatexbase.create_callback}
% |luatexbase.create_callback(|\meta{name},meta{type},\meta{default}|)|
% Defines a user defined callback. The last argument is a default
% functtion of |false|.
%
% \noindent
% \DescribeMacro{luatexbase.call_callback}
% |luatexbase.create_callback(|\meta{name},\ldots|)|
% Calls a user defined callback with the supplied arguments.
%
% \endgroup
%
% \StopEventually{}
%
% \section{Implementation}
%
%
% \subsection{Plain \TeX\ support}
%
%    \begin{macrocode}
%<*plain>
%    \end{macrocode}
%
%    \begin{macrocode}
\edef\etatcatcode{\the\catcode`\@}
\catcode`\@=11
%    \end{macrocode}
%
%    \begin{macrocode}
\ifx\e@alloc\@undefined\else
  \expandafter\endinput
\fi
%    \end{macrocode}
%
% \subsubsection{Fixes to \texttt{etex.src}/\texttt{etex.sty}}
%
% These could and probably should be made directly in an
% update to etex.src which already has some Lua\TeX{}-specific
% code, but does not define the correct range for Lua\TeX{}.
%
%    \begin{macrocode}
% 2015-07-13 higher range in luatex
\edef \et@xmaxregs {\ifx\directlua\@undefined 32768\else 65536\fi}
% luatex/xetex also allow more math fam
\edef \et@xmaxfam {\ifx\Umathchar\@undefined\sixt@@n\else\@cclvi\fi}
%    \end{macrocode}
%
%    \begin{macrocode}
\count 270=\et@xmaxregs % locally allocates \count registers 32767, 32766, ...
\count 271=\et@xmaxregs % ditto for \dimen registers
\count 272=\et@xmaxregs % ditto for \skip registers
\count 273=\et@xmaxregs % ditto for \muskip registers
\count 274=\et@xmaxregs % ditto for \box registers
\count 275=\et@xmaxregs % ditto for \toks registers
\count 276=\et@xmaxregs % ditto for \marks classes
%    \end{macrocode}
%
%    \begin{macrocode}
% and 256 or 16 fam
\outer\def\newfam{\alloc@8\fam\chardef\et@xmaxfam}
%    \end{macrocode}
%
% End of proposed changes to \texttt{etex.src}
%
% \subsubsection{luatex specific settings}
% Switch to global cf |luatex.sty| to leave room for inserts
% not really needed for Lua\TeX{} but possibly most compatible
% with existing use.
%    \begin{macrocode}
\let\newcount\globcount
\let\newdimen\globdimen
\let\newskip\globskip
\let\newbox\globbox
%    \end{macrocode}
%
% Define|\e@alloc| as in latex (the existing macros in |etex.src|
% hard to extend to further register types as they assume specific
% 26x and 27x count range. For compatibility the existing register
% allocation is not changed.
%
%    \begin{macrocode}
\chardef\e@alloc@top=65535
\let\e@alloc@chardef\chardef
%    \end{macrocode}
%
%    \begin{macrocode}
\def\e@alloc#1#2#3#4#5#6{%
  \global\advance#3\@ne
  \e@ch@ck{#3}{#4}{#5}#1%
  \allocationnumber#3\relax
  \global#2#6\allocationnumber
  \wlog{\string#6=\string#1\the\allocationnumber}}%
%    \end{macrocode}
%
%    \begin{macrocode}
\gdef\e@ch@ck#1#2#3#4{%
  \ifnum#1<#2\else
    \ifnum#1=#2\relax
      #1\@cclvi
      \ifx\count#4\advance#1 10 \fi
    \fi
    \ifnum#1<#3\relax
    \else
      \errmessage{No room for a new \string#4}%
    \fi
  \fi}%
%    \end{macrocode}
%
% Two simple \LaTeX\ macros used in |ltlatex.sty|.
%    \begin{macrocode}
\long\def \@gobble #1{}
\long\def\@firstofone#1{#1}
%    \end{macrocode}
%
%    \begin{macrocode}
% Fix up allocations not to clash with |etex.src|.
%    \end{macrocode}
%
%    \begin{macrocode}
\newcount\e@alloc@attribute@count
\newcount\e@alloc@ccodetable@count
\newcount\e@alloc@luafunction@count
\newcount\e@alloc@whatsit@count
%    \end{macrocode}
%
%    \begin{macrocode}
% Input the main code.
\input ltluatex.sty\relax
%    \end{macrocode}
%
% Reset the catcode of |@|.
%    \begin{macrocode}
\catcode`\@=\etatcatcode\relax
%    \end{macrocode}
%
%    \begin{macrocode}
%</plain>
%    \end{macrocode}
%
% \subsection{luatexbase interface}
%
%    \begin{macrocode}
%<*emu>
\edef\emuatcatcode{\the\catcode`\@}
\catcode`\@=11
%    \end{macrocode}
%
%    \begin{macrocode}
% emu-luatexbase.sty v 0.01 2015/07/14
% minimal luatexbase emulation, for testing with
% current luaotfload package
%    \end{macrocode}
%
%
%    \begin{macrocode}
\ifx\e@alloc\@undefined
%    \end{macrocode}
%
%    \begin{macrocode}
% In pre 2014 latex, or plain TeX load |etex.{sty,src}|
%    \end{macrocode}
%
%    \begin{macrocode}
  \ifx\documentclass\@undefined
    \ifx\loccount\@undefined
      \input{etex.src}%
    \fi
    \catcode`\@=11 %
  \else
    \RequirePackage{etex}
  \fi
%    \end{macrocode}
% Load the code for the plain interface (which also works in old
% \LaTeX).
%    \begin{macrocode}
  
\edef\etatcatcode{\the\catcode`\@}
\catcode`\@=11

\ifx\e@alloc\@undefined\else
\expandafter\endinput
\fi

%
% fixes to etex.src, 
% These couldand probably should be made directly in an
% update to etex.src which already has some luatex-specific
% code, but does not define the correct range for luatex.

% 2015-07-13 higher range in luatex
\edef \et@xmaxregs {\ifx\directlua\@undefined 32768\else 65536\fi}

\count 270=\et@xmaxregs % locally allocates \count registers 32767, 32766, ...
\count 271=\et@xmaxregs % ditto for \dimen registers
\count 272=\et@xmaxregs % ditto for \skip registers
\count 273=\et@xmaxregs % ditto for \muskip registers
\count 274=\et@xmaxregs % ditto for \box registers
\count 275=\et@xmaxregs % ditto for \toks registers
\count 276=\et@xmaxregs % ditto for \marks classes

%%%%%%%%%%%%%%%%%%%%%%%%%%%%%%%%%%%%%%%%%%%%%%%%%%%%%%




% define\e@alloc as in latex (the existing macros in etex.src
% hard to extend to further register types as they assume specific
% 26x and 27x count range. For compatibility the existing register
% allocation is not changed.


\def\e@alloc#1#2#3#4#5#6{%
  \global\advance#3\@ne
  \e@ch@ck{#3}{#4}{#5}#1%
  \allocationnumber#3\relax
  \global#2#6\allocationnumber
  \wlog{\string#6=\string#1\the\allocationnumber}}%
\gdef\e@ch@ck#1#2#3#4{%
  \ifnum#1<#2\else
    \ifnum#1=#2\relax
      #1\@cclvi
      \ifx\count#4\advance#1 10 \fi
    \fi
    \ifnum#1<#3\relax
    \else
      \errmessage{No room for a new #4}%
    \fi
  \fi}%


%%%%%%
\long\def \@gobble #1{}
\long\def\@firstofone#1{#1}
\def\makeatother{\catcode`\@12\relax}
\input ltluatex.sty


% fix up allocations not to clash with etex.src
% \count registers 256-259 and 267-269 are not (yet) used
% 

% ltluatex uses 258 and 259
% also uses 260 and 261 but change them to 267 and 268
\def\newluafunction{%
  \e@alloc\luafunction\e@alloc@chardef
    {\count267}\m@ne\e@alloc@top
}
\count267=\z@
\def\newwhatsit#1{%
  \e@alloc\whatsit\e@alloc@chardef
    {\count268}\m@ne\e@alloc@top#1%
}
\count268=\z@

\directlua{
local function new_whatsit(name)
  tex_setcount("global", 267, tex_count[267] + 1)
  if tex_count[267] > 65534 then
    latex_error("No room for a new custom whatsit")
    return -1
  end
  texio_write_nl("Custom whatsit " .. name .. " = " .. tex_count[267])
  return tex_count[267]
end
latex.new_whatsit = new_whatsit
}

\catcode`\@=\etatcatcode\relax
%
\else
%    \end{macrocode}
% In \LaTeX\ releases from 2015/01/01 load |ltluatex|,
% load |ltluatex.sty| unless things already defined.
%    \begin{macrocode}
  \ifx\catcodetable@initex\@undefined
    \ifx\RequirePackage\@undefined
      
\edef\etatcatcode{\the\catcode`\@}
\catcode`\@=11

\ifx\e@alloc\@undefined\else
\expandafter\endinput
\fi

%
% fixes to etex.src, 
% These couldand probably should be made directly in an
% update to etex.src which already has some luatex-specific
% code, but does not define the correct range for luatex.

% 2015-07-13 higher range in luatex
\edef \et@xmaxregs {\ifx\directlua\@undefined 32768\else 65536\fi}

\count 270=\et@xmaxregs % locally allocates \count registers 32767, 32766, ...
\count 271=\et@xmaxregs % ditto for \dimen registers
\count 272=\et@xmaxregs % ditto for \skip registers
\count 273=\et@xmaxregs % ditto for \muskip registers
\count 274=\et@xmaxregs % ditto for \box registers
\count 275=\et@xmaxregs % ditto for \toks registers
\count 276=\et@xmaxregs % ditto for \marks classes

%%%%%%%%%%%%%%%%%%%%%%%%%%%%%%%%%%%%%%%%%%%%%%%%%%%%%%




% define\e@alloc as in latex (the existing macros in etex.src
% hard to extend to further register types as they assume specific
% 26x and 27x count range. For compatibility the existing register
% allocation is not changed.


\def\e@alloc#1#2#3#4#5#6{%
  \global\advance#3\@ne
  \e@ch@ck{#3}{#4}{#5}#1%
  \allocationnumber#3\relax
  \global#2#6\allocationnumber
  \wlog{\string#6=\string#1\the\allocationnumber}}%
\gdef\e@ch@ck#1#2#3#4{%
  \ifnum#1<#2\else
    \ifnum#1=#2\relax
      #1\@cclvi
      \ifx\count#4\advance#1 10 \fi
    \fi
    \ifnum#1<#3\relax
    \else
      \errmessage{No room for a new #4}%
    \fi
  \fi}%


%%%%%%
\long\def \@gobble #1{}
\long\def\@firstofone#1{#1}
\def\makeatother{\catcode`\@12\relax}
\input ltluatex.sty


% fix up allocations not to clash with etex.src
% \count registers 256-259 and 267-269 are not (yet) used
% 

% ltluatex uses 258 and 259
% also uses 260 and 261 but change them to 267 and 268
\def\newluafunction{%
  \e@alloc\luafunction\e@alloc@chardef
    {\count267}\m@ne\e@alloc@top
}
\count267=\z@
\def\newwhatsit#1{%
  \e@alloc\whatsit\e@alloc@chardef
    {\count268}\m@ne\e@alloc@top#1%
}
\count268=\z@

\directlua{
local function new_whatsit(name)
  tex_setcount("global", 267, tex_count[267] + 1)
  if tex_count[267] > 65534 then
    latex_error("No room for a new custom whatsit")
    return -1
  end
  texio_write_nl("Custom whatsit " .. name .. " = " .. tex_count[267])
  return tex_count[267]
end
latex.new_whatsit = new_whatsit
}

\catcode`\@=\etatcatcode\relax
%
    \else
      \RequirePackage{ltluatex}
    \fi
  \fi
\fi
%    \end{macrocode}
%
% Simple require wrapper so that |luaotfload| does get loaded.
%    \begin{macrocode}
\def\RequireLuaModule#1{\directlua{require("#1")}}
%    \end{macrocode}
%
% Extended catcode table support.  Use the names from the previous
% \textsf{luatexbase} and \textsf{luatex} packages.
%    \begin{macrocode}
\let\CatcodeTableIniTeX\catcodetable@initex
\let\CatcodeTableString\catcodetable@string
\let\CatcodeTableLaTeX\catcodetable@latex
\let\CatcodeTableLaTeXAtLetter\catcodetable@atletter
%    \end{macrocode}
%
% Additional tables declared in the previous interface.
%    \begin{macrocode}
\newcatcodetable\CatcodeTableOther
\setcatcodetable\CatcodeTableOther{%
  \catcodetable\CatcodeTableString
  \catcode32 12 }
%    \end{macrocode}
%
%    \begin{macrocode}
\newcatcodetable\CatcodeTableExpl
\setcatcodetable\CatcodeTableExpl{%
  \catcodetable\CatcodeTableLaTeX
  \catcode126 10 % tilde is a space char
  \catcode32  9  % space is ignored
  \catcode9   9  % tab also ignored
  \catcode95  11 % underscore letter
  \catcode58  11 % colon letter
}
%    \end{macrocode}
%
% Top level access to catcodetable stack.
%    \begin{macrocode}
\def\BeginCatcodeRegime#1{%
  \@pushcatcodetable
  \catcodetable#1\relax}
%    \end{macrocode}
%
%    \begin{macrocode}
\def\EndCatcodeRegime{%
  \@popcatcodetable}
%    \end{macrocode}
%
% The implementation of the stack is completely
% different, but usage should match.
%    \begin{macrocode}
\let\PushCatcodeTableNumStack\@pushcatcodetable
\let\PopCatcodeTableNumStack\@popcatcodetable
%    \end{macrocode}
%
% \subsubsection{Additional lua code}
%    \begin{macrocode}
\directlua{
%    \end{macrocode}
%
% Remove all registered callbacks, then disable.
% Set to false if optional second argument is |true|.
%    \begin{macrocode}
function luatexbase.reset_callback(name,make_false)
  for _,v in pairs(luatexbase.callback_descriptions(name))
  do
    luatexbase.remove_from_callback(name,v)
  end
  if make_false == true then
    luatexbase.disable_callback(name)
  end
end
%    \end{macrocode}
%
% Emulate the catcodetables table.
% Explicitly fill the table rather than rely on the metatable call to
% |registernumber| as that is unreliable on old Lua\TeX{}.
%    \begin{macrocode}
luatexbase.catcodetables=setmetatable(
 {['latex-package'] = \number\catcodetable@atletter,
  ini    = \number\CatcodeTableIniTeX,
  string = \number\CatcodeTableString,
  other  = \number\CatcodeTableOther,
  latex  = \number\catcodetable@latex,
  expl   = \number\CatcodeTableExpl,
  expl3  = \number\CatcodeTableExpl},
 { __index = function(t,key)
    return luatexbase.registernumber(key) or nil
  end}
)}
%    \end{macrocode}
%
% On old Lua\TeX{} workaround hashtable issues.
% Allocate in \TeX{}, and also directly add to |luatexbase.catcodetables|.
%    \begin{macrocode}
\ifnum\luatexversion<80 %
\def\newcatcodetable#1{%
  \e@alloc\catcodetable\chardef
    \e@alloc@ccodetable@count\m@ne{"8000}#1%
  \initcatcodetable\allocationnumber
  {\escapechar=\m@ne
  \directlua{luatexbase.catcodetables['\string#1']=%
    \the\allocationnumber}}%
}
\fi
%    \end{macrocode}
%
%    \begin{macrocode}
\directlua{
%    \end{macrocode}
%
% |priority_in_callback| returns position in the callback list.
% Not provided by default in |ltluatex| as usually it is just used 
% as a boolean test, for which |in_callback| is provided.
%    \begin{macrocode}
function luatexbase.priority_in_callback (name,description)
  for i,v in ipairs(luatexbase.callback_descriptions(name))
  do
    if v == description then
      return i
    end
  end
  return false
end
%    \end{macrocode}
% 
% The (unreleased) version 0.7 of \textsf{luatexbase} provided this
% boolean test under a different name, so we provide an alias here.
%    \begin{macrocode}
luatexbase.is_active_callback = luatexbase.in_callback
%    \end{macrocode}
%
% |ltluatex| implementation of |provides_module| does not return
% print functions so define modified version here.
%    \begin{macrocode}
luatexbase.base_provides_module=luatexbase.provides_module
function luatexbase.provides_module(info)
  luatexbase.base_provides_module(info)
  return
    function(s) return luatexbase.module_error(info.name, s) end,
    function(s) return luatexbase.module_warning(info.name, s) end,
    function(s) return luatexbase.module_info(info.name, s) end,
    function(s) return luatexbase.module_info(info.name, s) end
end
}
%    \end{macrocode}
%
% Same for attribute table as catcode tables. In old Lua\TeX{}, add to the
% |luatexbase.attributes| table directly.
%    \begin{macrocode}
\ifnum\luatexversion<80 %
\def\attribute#1{%
  \e@alloc\attribute\attributedef
    \e@alloc@attribute@count\m@ne\e@alloc@top#1%
  {\escapechar=\m@ne
  \directlua{luatexbase.attributes['\string#1']=%
    \the\allocationnumber}}%
}
\fi
%    \end{macrocode}
%
% Resolve name clashes and prefixed name issues.
%    \begin{macrocode}
\let\newluatexattribute\newattribute
\let\setluatexattribute\setattribute
\let\unsetluatexattribute\unsetattribute
\let\newluatexcatcodetable\newcatcodetable
\let\setluatexcatcodetable\setcatcodetable
\let\luatexbase@directlua\directlua
\let\luatebase@ensure@primitive\@gobble
%    \end{macrocode}
%
% reset catcode of |@|.
%    \begin{macrocode}
\catcode`\@=\emuatcatcode\relax
%    \end{macrocode}
%
%    \begin{macrocode}
%</emu>
%    \end{macrocode}
%
%    \begin{macrocode}
%<*emu-cmp,emu-mod,emu-loa,emu-reg,emu-att,emu-cct,emu-mcb>
%    \end{macrocode}
%
%    \begin{macrocode}
\ifx\RequirePackage\undefined
  % \iffalse meta-comment
%
% Copyright 2015
% The LaTeX3 Project and any individual authors listed elsewhere
% in this file.
%
% It may be distributed and/or modified under the conditions of
% the LaTeX Project Public License (LPPL), either version 1.3c of
% this license or (at your option) any later version.  The latest
% version of this license is in the file:
%
%   http://www.latex-project.org/lppl.txt
%
%
%
%<emu>\ifx\BeginCatcodeRegime\undefined\else\expandafter\endinput\fi
%<plain>\ifx\newluafunction\undefined\else\expandafter\endinput\fi
%<tex,sty>\ifx
%<sty>  \ProvidesPackage\undefined\begingroup\def\ProvidesPackage
%<tex>  \ProvidesFile\undefined\begingroup\def\ProvidesFile
%<tex,sty>  #1#2[#3]{\endgroup\immediate\write16{Package: #1 #3}}
%<tex,sty>\fi
%<plain>\ProvidesFile{ltluatex.tex}
%<emu>\ProvidesPackage{luatexbase}
%<emu-cmp>\ProvidesPackage{luatexbase-compat}
%<emu-mod>\ProvidesPackage{luatexbase-modutils}
%<emu-loa>\ProvidesPackage{luatexbase-loader}
%<emu-reg>\ProvidesPackage{luatexbase-regs}
%<emu-att>\ProvidesPackage{luatexbase-attr}
%<emu-cct>\ProvidesPackage{luatexbase-cctb}
%<emu-mcb>\ProvidesPackage{luatexbase-mcb}
%<*driver>
\ProvidesFile{luatexbase.dtx}
%</driver>
%<*tex,sty>
[2015/08/01 v1.0a
%</tex,sty>
%<plain>  luatex support for plain TeX (core + catcode stack)
%<emu>  luatexbase interface to luatex
%<emu-cmp>  luatexbase interface to luatex (legacy subpackage)
%<emu-mod>  luatexbase interface to luatex (legacy subpackage)
%<emu-loa>  luatexbase interface to luatex (legacy subpackage)
%<emu-reg>  luatexbase interface to luatex (legacy subpackage)
%<emu-cct>  luatexbase interface to luatex (legacy subpackage)
%<emu-mcb>  luatexbase interface to luatex (legacy subpackage)
%<*tex,sty>
]
%</tex,sty>
%<*driver>
\documentclass{ltxdoc}
\GetFileInfo{luatexbase.dtx}
\begin{document}
\title{\filename\\(Lua\TeX{}-specific support, luatexbase interface)}
\author{David Carlisle and Joseph Wright\footnote{Significant portions
  of the code here are adapted/simplified from the packages \textsf{luatex} and
  \textsf{luatexbase} written by Heiko Oberdiek, \'{E}lie Roux,
  Manuel P\'{e}gouri\'{e}-Gonnar and Philipp Gesang.}}
\date{\filedate}
\maketitle
\setcounter{tocdepth}{2}
\tableofcontents
\DocInput{\filename}
\end{document}
%</driver>
% \fi
%
% \CheckSum{0}
%
% \section{Overview}
%
% Lua\TeX{} adds a number of engine-specific functions to \TeX{}. Several of
% these require set up that is best done in the kernel or need related support
% functions. This file provides \emph{basic} support for Lua\TeX{} at the
% \LaTeXe{} kernel level plus as a loadable package which can be used with plain
% \TeX{} and \LaTeX{}. For \LaTeX{}, the package adds a small number of commands
% which do not fit into the scope of the kernel.
%
% This file contains code for both \TeX{} (to be stored as part of the format)
% and Lua (to be loaded at the start of each job). In the Lua code, the kernel
% uses the namespace |latex| which is \emph{reserved for use by the team}.
%
% The following |\count| registers are used here for register allocation:
% \begin{itemize}
%  \item[\texttt{\string\e@alloc@attribute@count}] Attributes (default 258)
%  \item[\texttt{\string\e@alloc@ccodetable@count}] Category code tables (default 259)
%  \item[\texttt{\string\e@alloc@luafunction@count}] Lua functions (default 260)
%  \item[\texttt{\string\e@alloc@whatsit@count}] User whatsits (default 261)
% \end{itemize}
% (|\count256| is used for |\newmarks| allocation and |\count257|
% is used for\break
% |\newXeTeXintercharclass| with Xe\TeX{}, with code defined in
% \texttt{ltfinal.dtx}).
% With any \LaTeXe{} kernel from 2015 onward these registers are part of
% the block in the extended area reserved by the kernel (prior to 2015 the
% \LaTeXe{} kernel did not provide any functionality for the extended
% allocation area).
%
% \section{Core \TeX{} functionality}
%
% The commands defined here are defined in |ltluatex.def| for
% possible inclusion in a future \LaTeX\ format, however for ease of
% testing they are also currently defined in the package file
% |ltluatex.sty| along with the additional commands described in the
% next section.
%
% \DescribeMacro{\newattribute}
% |\newattribute{|\meta{attribute}|}|\\
% Defines a named \cs{attribute}, indexed from~$1$
% (\emph{i.e.}~|\attribute0| is never defined). Attributes initially
% have the marker value |-"7FFFFFFF| (`unset') set by the engine.
%
% \noindent
% \DescribeMacro{\newcatcodetable}
% |\newcatcodetable{|\meta{catcodetable}|}|\\
% Defines a named \cs{catcodetable}, indexed from~$1$
% (|\catcodetable0| is never assigned). A new catcode table will be
% populated with exactly those values assigned by Ini\TeX{} (as described
% in the Lua\TeX{} manual).
%
% \noindent
% \DescribeMacro{\newluafunction}
% |\newluafunction{|\meta{function}|}|\\
% Defines a named \cs{luafunction}, indexed from~$1$ (Lua indexes from
% $1$ so |\luafunction0| is not available).
%
% \noindent
% \DescribeMacro{\newwhatsit}
% |\newwhatsit{|\meta{whatsit}|}|\\
% Defines a custom \cs{whatsit}, indexed from~$1$.
%
% \noindent
% \DescribeMacro{\catcodetable@initex}
% \DescribeMacro{\catcodetable@string}
% \DescribeMacro{\catcodetable@latex}
% \DescribeMacro{\catcodetable@atletter}
% Predefined category code tables with the obvious assignments. Note
% that the |latex| and |atletter| tables set the full Unicode range
% to the codes predefined by the kernel.
%
% \section{The \texttt{ltluatex} Package \TeX{} functionality}
%
% \noindent
% \DescribeMacro{\setattribute}
% \DescribeMacro{\unsetattribute}
% |\setattribute{|\meta{attribute}|}{|\meta{value}|}|\\
% |\unsetattribute{|\meta{attribute}|}|\\
% Set and unset attributes in a manner analogous to |\setlength|. Note that
% attributes take a marker value when unset so this operation is distinct
% from setting the value to zero.
%
% \noindent
% \DescribeMacro{\setcatcodetable}
% |\setcatcodetable{|\meta{table}|}{|\meta{catcodes}|}|\\
% Sets the \meta{table} (which must have been previously defined) to
% apply the \meta{catcodes} specified.
%
% \noindent
% \DescribeMacro{\setrangecatcode}
% |\setrangecatcode{|\meta{start}|}{|\meta{end}|}{|\meta{catcode}|}|\\
% Sets all characters in the range \meta{start}--\meta{end} inclusive to
% have the \meta{catcode} specified.
%
% \noindent
% \DescribeMacro{\@pushcatcodes}
% \DescribeMacro{\@popcatcodes}
% |\@pushcatcodetable|\\
% |\@popcatcodetable|\\
% This pair of commands enable the current category code r\'{e}gime to
% be saved and restored meaning that arbitrary catcode changes can be made.
% This functionality will normally be used in concert with applying
% catcode tables. For example
% \begin{verbatim}
% \catcode`\Z=4 %
% \@pushcatcodetable
% \catcodetable\catcodetable@latex
% % Code here
% \@popcatcodetable
% \showthe\catcode`\Z
% \end{verbatim}
% will ensure that the `content' is set with normal category codes but
% allow restoration of the non-standard codes at the conclusion. Importantly,
% it does not require that anything is known about the catcode situation in
% advance (\emph{cf.}~a more traditional approach to saving the state of
% targeting characters).
%
% \section{The \texttt{luatexbase} package interface}
%
% The interfaces defined here are closely modelled on the earlier
% \textsf{luatexbase} package, which in turn copies parts of the
% \textsf{luatex} package.
% A closer emulation of the existing \textsf{luatexbase} is provided by the
% package currently called |emu-luatexbase|. Having a different name
% allows both packages to be easily used while testing, although
% possibly |emu-luatexbase| can be the source of an updated
% \textsf{luatexbase} package, depending on how things turn out.
%
% \subsection{\TeX}
% \subsubsection{Alternative catcode table handling}
%
% \noindent
% \DescribeMacro{\CatcodeTableIniTeX}
% \DescribeMacro{\CatcodeTableString}
% \DescribeMacro{\CatcodeTableLaTeX}
% \DescribeMacro{\CatcodeTableLaTeXAtLetter}
% \DescribeMacro{\CatcodeTableOther}
% \DescribeMacro{\CatcodeTableExpl}
% Alternative names for some predefined catcodetables, and some
% additional ones are declared.
%
% \noindent
% \DescribeMacro{\BeginCatcodeRegime}
% \DescribeMacro{\EndCatcodeRegime}
% A simple wrapper around |\@pushcatcodetable| providing aslightly
% different interface.
%
% \noindent
% \DescribeMacro{\PushCatcodeTableNumStack}
% \DescribeMacro{\PopCatcodeTableNumStack}
% These are defined to be aliases for |\@pushcatcodetable| and
% |\@popcatcodetable| although the actual implemenatation is quite different
% to the older packages, the use of the commands should match.
%
% \subsection{Lua}
% \subsubsection{predefined catcode table names}
% As in the previous packages the |catcodetables| array is extended
% with aliases for the predefined tables so you can use |catcodetables.expl| in
% addition to |catcodetables.CatcodeTableExpl|.
%
% \subsubsection{callback queries}
% In the earlier interfaces the function
% |luatexbase.priority_in_callback|
% is provided to return a number indicating th eposition of a
% specified function in a callback list. However it is usually used
% just as a boolean test that the function is registered with the
% callback. |ltluatex| does not expose the priority numbers so this
% function is defined as an alias to the function
% |luatexbase.in_callback| which returns a boolean.
%
% \subsubsection{Module declaration}
% The |luatexbase| version of |provides_module| returns a list of log
% and error functions so that it is usually called as:\\
% |local err, warning, info, log = luatexbase.provides_module({name=..|
%
% \section{Plain \TeX\ interface}
%
% The |ltluatex| interface may be used with plain \TeX\ using 
% |\input ltluatex| this inputs |ltluatex.tex| which inputs |etex.src|
% if it is not already input, and then defines some internal commands to
% allow the \emph{extended} interafce defined by |ltluatex.sty| to be
% defined.
%
% The `luatexbase` package interface may also be used in plain \TeX,
% as before by simply inputting the package, currently |\input emu-luatexbase.sty|.
%
% \section{Lua functionality}
%
% \begingroup
%
% \begingroup\lccode`~=`_
% \lowercase{\endgroup\let~}_
% \catcode`_=12
%
% \section{Allocators in Lua}
%
% \DescribeMacro{luatexbase.new_attribute}
% |luatexbase.new_attribute(|\meta{attribute}|)|\\
% Returns an allocation number for the \meta{attribute}, indexed from~$1$.
% The attribute will be initialised with the marker value |-"7FFFFFFF|
% (`unset'). The attribute allocation sequence is shared with the \TeX{}
% code but this function does \emph{not} define a token using
% |\attributedef|.
% The attribute name is recorded in the |attributes| table. A
% metatable is provided so that the table syntax can be used
% consistently for attributes declared in \TeX\ or lua.
%
% \noindent
% \DescribeMacro{luatexbase.new_whatsit}
% |luatexbase.new_whatsit(|\meta{whatsit}|)|\\
% Returns an allocation number for the custom \meta{whatsit}, indexed from~$1$.
%
% \noindent
% \DescribeMacro{luatexbase.new_bytecode}
% |luatexbase.new_bytecode(|\meta{bytecode}|)|\\
% Returns an allocation number for a bytecode register, indexed from~$1$.
% The optional \meta{name} argument is just used for logging.
%
% \noindent
% \DescribeMacro{luatexbase.new_chunkname}
% |luatexbase.new_chunkname(|\meta{chunkname}|)|\\
% Returns an allocation number for a lua chunk name for use with 
% |\directlua| and |\latelua|, indexed from~$1$.
% The number is returned and also \meta{name} argument is added to the
% |lua.name| array at that index.
%
%
% \section{Lua access to \TeX{} register numbers}
%
% \DescribeMacro{luatexbase.registernumber}
% |luatexbase.registernumer(|\meta{name}|)|\\
% Sometimes (notably in the case of Lua attributes) it is necessary to
% access a register \emph{by number} that has been allocated by \TeX{}.
% This package provides a function to look up the relevant number
% using Lua\TeX{}'s internal tables. So after for example
% |\newattribute\myattrib|, |\myattrib| would be defined by (say)
% |\myattrib=\attribute15|. The function call |luatexbase.registernumer("myattrib")|
% will then return the register number, $15$ in this case. If the string passed
% as argument does not correspond to a token defined by |\attributedef|,
% |\countdef| or similar commands, the Lua value |false| is returned.
%
% As an example, consider the input:
%\begin{verbatim}
% \newcommand\test[1]{%
% \typeout{#1: \expandafter\meaning\csname#1\endcsname^^J
% \space\space\space\space
% \directlua{tex.write(luatexbase.registernumber("#1") or "bad input")}%
% }}
%
% \test{undefinedrubbish}
%
% \test{space}
%
% \test{hbox}
%
% \test{@MM}
%
% \test{@tempdima}
% \test{@tempdimb}
%
% \test{strutbox}
%
% \test{sixt@@n}
%
% \attrbutedef\myattr=12
% \myattr=200
% \test{myattr}
%
%\end{verbatim}
%
% If the demonstration code is processed with Lua\LaTeX{} then the following
% would be produced in the log and terminal output.
%\begin{verbatim}
% undefinedrubbish: \relax
%      bad input
% space: macro:->
%      bad input
% hbox: \hbox
%      bad input
% @MM: \mathchar"4E20
%      20000
% @tempdima: \dimen14
%      14
% @tempdimb: \dimen15
%      15
% strutbox: \char"B
%      11
% sixt@@n: \char"10
%      16
% myattr: \attribute12
%      12
%\end{verbatim}
%
% Notice how undefined commands, or commands unrelated to registers
% do not produce an error, just return |false| and so print
% |bad input| here. Note also that commands defined by |\newbox| work and
% return the number of the box register even though the actual command
% holding this number is a |\chardef| defined token (there is no
% |\boxdef|).
%
% \subsection{Module utilities}
%
% \DescribeMacro{luatexbase.provides_module}
% |luatexbase.provides_module(|\meta{info}|)|\\
% This function is used by modules to identify themselves; the |info| should be
% a table containing information about the module. The required field
% |name| must contain the name of the module. It is recommended to provide a
% field |date| in the usual \LaTeX{} format |yyyy/mm/dd|. Optional fields
% |version| (a string) and |description| may be used if present. This
% information will be recorded in the log. Other fields are ignored.
%
% \noindent
% \DescribeMacro{luatexbase.module_info}
% \DescribeMacro{luatexbase.module_warning}
% \DescribeMacro{luatexbase.module_error}
% |luatexbase.module_info(|\meta{module}, \meta{text}|)|\\
% |luatexbase.module_warning(|\meta{module}, \meta{text}|)|\\
% |luatexbase.module_error(|\meta{module}, \meta{text}|)|\\
% These functions are similar to \LaTeX{}'s |\PackageError|, |\PackageWarning|
% and |\PackageInfo| in the way they format the output.  No automatic line
% breaking is done, you may still use |\n| as usual for that, and the name of
% the package will be prepended to each output line.
%
% Note that |luatexbase.module_error| raises an actual Lua error with |error()|,
% which currently means a call stack will be dumped. While this may not
% look pretty, at least it provides useful information for tracking the
% error down.
%
% \subsection{Callback management}
%
% \noindent
% \DescribeMacro{luatexbase.add_to_callback}
% |luatexbase.add_to_callback(|^^A
% \meta{callback}, \meta{function}, \meta{description}|)|
% Registers the \meta{function} into the \meta{callback} with a textual
% \meta{description} of the function. Functions are inserted into the callback
% in the order loaded.
%
% \noindent
% \DescribeMacro{luatexbase.remove_from_callback}
% |luatexbase.remove_from_callback(|\meta{callback}, \meta{description}|)|
% Removes the function with \meta{description} from the \meta{callback}.
% The removed function and its description 
% are returned as the results of this function.
%
% \noindent
% \DescribeMacro{luatexbase.in_callback}
% |luatexbase.in_callback(|\meta{callback}, \meta{description}|)|
% Checks if the \meta{description} matches one of the functions added
% to the list for the \meta{callback}, returning a boolean value.
%
% \noindent
% \DescribeMacro{luatexbase.disable_callback}
% |luatexbase.disable_callback(|\meta{callback}|)|
% Sets the \meta{callback} to \texttt{false} as described in the Lua\TeX{}
% manual for the underlying \texttt{callback.register} built-in. Callbacks
% will only be set to false (and thus be skipped entirely) if there are
% no functions registered using the callback.
%
% \noindent
% \DescribeMacro{luatexbase.callback_descriptions}
% A list of the descriptions of functions registered to the specified
% callback is returned. |{}| is returned if there are no functions registered.
%
% \noindent
% \DescribeMacro{luatexbase.create_callback}
% |luatexbase.create_callback(|\meta{name},meta{type},\meta{default}|)|
% Defines a user defined callback. The last argument is a default
% functtion of |false|.
%
% \noindent
% \DescribeMacro{luatexbase.call_callback}
% |luatexbase.create_callback(|\meta{name},\ldots|)|
% Calls a user defined callback with the supplied arguments.
%
% \endgroup
%
% \StopEventually{}
%
% \section{Implementation}
%
%
% \subsection{Plain \TeX\ support}
%
%    \begin{macrocode}
%<*plain>
%    \end{macrocode}
%
%    \begin{macrocode}
\edef\etatcatcode{\the\catcode`\@}
\catcode`\@=11
%    \end{macrocode}
%
%    \begin{macrocode}
\ifx\e@alloc\@undefined\else
  \expandafter\endinput
\fi
%    \end{macrocode}
%
% \subsubsection{Fixes to \texttt{etex.src}/\texttt{etex.sty}}
%
% These could and probably should be made directly in an
% update to etex.src which already has some Lua\TeX{}-specific
% code, but does not define the correct range for Lua\TeX{}.
%
%    \begin{macrocode}
% 2015-07-13 higher range in luatex
\edef \et@xmaxregs {\ifx\directlua\@undefined 32768\else 65536\fi}
% luatex/xetex also allow more math fam
\edef \et@xmaxfam {\ifx\Umathchar\@undefined\sixt@@n\else\@cclvi\fi}
%    \end{macrocode}
%
%    \begin{macrocode}
\count 270=\et@xmaxregs % locally allocates \count registers 32767, 32766, ...
\count 271=\et@xmaxregs % ditto for \dimen registers
\count 272=\et@xmaxregs % ditto for \skip registers
\count 273=\et@xmaxregs % ditto for \muskip registers
\count 274=\et@xmaxregs % ditto for \box registers
\count 275=\et@xmaxregs % ditto for \toks registers
\count 276=\et@xmaxregs % ditto for \marks classes
%    \end{macrocode}
%
%    \begin{macrocode}
% and 256 or 16 fam
\outer\def\newfam{\alloc@8\fam\chardef\et@xmaxfam}
%    \end{macrocode}
%
% End of proposed changes to \texttt{etex.src}
%
% \subsubsection{luatex specific settings}
% Switch to global cf |luatex.sty| to leave room for inserts
% not really needed for Lua\TeX{} but possibly most compatible
% with existing use.
%    \begin{macrocode}
\let\newcount\globcount
\let\newdimen\globdimen
\let\newskip\globskip
\let\newbox\globbox
%    \end{macrocode}
%
% Define|\e@alloc| as in latex (the existing macros in |etex.src|
% hard to extend to further register types as they assume specific
% 26x and 27x count range. For compatibility the existing register
% allocation is not changed.
%
%    \begin{macrocode}
\chardef\e@alloc@top=65535
\let\e@alloc@chardef\chardef
%    \end{macrocode}
%
%    \begin{macrocode}
\def\e@alloc#1#2#3#4#5#6{%
  \global\advance#3\@ne
  \e@ch@ck{#3}{#4}{#5}#1%
  \allocationnumber#3\relax
  \global#2#6\allocationnumber
  \wlog{\string#6=\string#1\the\allocationnumber}}%
%    \end{macrocode}
%
%    \begin{macrocode}
\gdef\e@ch@ck#1#2#3#4{%
  \ifnum#1<#2\else
    \ifnum#1=#2\relax
      #1\@cclvi
      \ifx\count#4\advance#1 10 \fi
    \fi
    \ifnum#1<#3\relax
    \else
      \errmessage{No room for a new \string#4}%
    \fi
  \fi}%
%    \end{macrocode}
%
% Two simple \LaTeX\ macros used in |ltlatex.sty|.
%    \begin{macrocode}
\long\def \@gobble #1{}
\long\def\@firstofone#1{#1}
%    \end{macrocode}
%
%    \begin{macrocode}
% Fix up allocations not to clash with |etex.src|.
%    \end{macrocode}
%
%    \begin{macrocode}
\newcount\e@alloc@attribute@count
\newcount\e@alloc@ccodetable@count
\newcount\e@alloc@luafunction@count
\newcount\e@alloc@whatsit@count
%    \end{macrocode}
%
%    \begin{macrocode}
% Input the main code.
\input ltluatex.sty\relax
%    \end{macrocode}
%
% Reset the catcode of |@|.
%    \begin{macrocode}
\catcode`\@=\etatcatcode\relax
%    \end{macrocode}
%
%    \begin{macrocode}
%</plain>
%    \end{macrocode}
%
% \subsection{luatexbase interface}
%
%    \begin{macrocode}
%<*emu>
\edef\emuatcatcode{\the\catcode`\@}
\catcode`\@=11
%    \end{macrocode}
%
%    \begin{macrocode}
% emu-luatexbase.sty v 0.01 2015/07/14
% minimal luatexbase emulation, for testing with
% current luaotfload package
%    \end{macrocode}
%
%
%    \begin{macrocode}
\ifx\e@alloc\@undefined
%    \end{macrocode}
%
%    \begin{macrocode}
% In pre 2014 latex, or plain TeX load |etex.{sty,src}|
%    \end{macrocode}
%
%    \begin{macrocode}
  \ifx\documentclass\@undefined
    \ifx\loccount\@undefined
      \input{etex.src}%
    \fi
    \catcode`\@=11 %
  \else
    \RequirePackage{etex}
  \fi
%    \end{macrocode}
% Load the code for the plain interface (which also works in old
% \LaTeX).
%    \begin{macrocode}
  
\edef\etatcatcode{\the\catcode`\@}
\catcode`\@=11

\ifx\e@alloc\@undefined\else
\expandafter\endinput
\fi

%
% fixes to etex.src, 
% These couldand probably should be made directly in an
% update to etex.src which already has some luatex-specific
% code, but does not define the correct range for luatex.

% 2015-07-13 higher range in luatex
\edef \et@xmaxregs {\ifx\directlua\@undefined 32768\else 65536\fi}

\count 270=\et@xmaxregs % locally allocates \count registers 32767, 32766, ...
\count 271=\et@xmaxregs % ditto for \dimen registers
\count 272=\et@xmaxregs % ditto for \skip registers
\count 273=\et@xmaxregs % ditto for \muskip registers
\count 274=\et@xmaxregs % ditto for \box registers
\count 275=\et@xmaxregs % ditto for \toks registers
\count 276=\et@xmaxregs % ditto for \marks classes

%%%%%%%%%%%%%%%%%%%%%%%%%%%%%%%%%%%%%%%%%%%%%%%%%%%%%%




% define\e@alloc as in latex (the existing macros in etex.src
% hard to extend to further register types as they assume specific
% 26x and 27x count range. For compatibility the existing register
% allocation is not changed.


\def\e@alloc#1#2#3#4#5#6{%
  \global\advance#3\@ne
  \e@ch@ck{#3}{#4}{#5}#1%
  \allocationnumber#3\relax
  \global#2#6\allocationnumber
  \wlog{\string#6=\string#1\the\allocationnumber}}%
\gdef\e@ch@ck#1#2#3#4{%
  \ifnum#1<#2\else
    \ifnum#1=#2\relax
      #1\@cclvi
      \ifx\count#4\advance#1 10 \fi
    \fi
    \ifnum#1<#3\relax
    \else
      \errmessage{No room for a new #4}%
    \fi
  \fi}%


%%%%%%
\long\def \@gobble #1{}
\long\def\@firstofone#1{#1}
\def\makeatother{\catcode`\@12\relax}
\input ltluatex.sty


% fix up allocations not to clash with etex.src
% \count registers 256-259 and 267-269 are not (yet) used
% 

% ltluatex uses 258 and 259
% also uses 260 and 261 but change them to 267 and 268
\def\newluafunction{%
  \e@alloc\luafunction\e@alloc@chardef
    {\count267}\m@ne\e@alloc@top
}
\count267=\z@
\def\newwhatsit#1{%
  \e@alloc\whatsit\e@alloc@chardef
    {\count268}\m@ne\e@alloc@top#1%
}
\count268=\z@

\directlua{
local function new_whatsit(name)
  tex_setcount("global", 267, tex_count[267] + 1)
  if tex_count[267] > 65534 then
    latex_error("No room for a new custom whatsit")
    return -1
  end
  texio_write_nl("Custom whatsit " .. name .. " = " .. tex_count[267])
  return tex_count[267]
end
latex.new_whatsit = new_whatsit
}

\catcode`\@=\etatcatcode\relax
%
\else
%    \end{macrocode}
% In \LaTeX\ releases from 2015/01/01 load |ltluatex|,
% load |ltluatex.sty| unless things already defined.
%    \begin{macrocode}
  \ifx\catcodetable@initex\@undefined
    \ifx\RequirePackage\@undefined
      
\edef\etatcatcode{\the\catcode`\@}
\catcode`\@=11

\ifx\e@alloc\@undefined\else
\expandafter\endinput
\fi

%
% fixes to etex.src, 
% These couldand probably should be made directly in an
% update to etex.src which already has some luatex-specific
% code, but does not define the correct range for luatex.

% 2015-07-13 higher range in luatex
\edef \et@xmaxregs {\ifx\directlua\@undefined 32768\else 65536\fi}

\count 270=\et@xmaxregs % locally allocates \count registers 32767, 32766, ...
\count 271=\et@xmaxregs % ditto for \dimen registers
\count 272=\et@xmaxregs % ditto for \skip registers
\count 273=\et@xmaxregs % ditto for \muskip registers
\count 274=\et@xmaxregs % ditto for \box registers
\count 275=\et@xmaxregs % ditto for \toks registers
\count 276=\et@xmaxregs % ditto for \marks classes

%%%%%%%%%%%%%%%%%%%%%%%%%%%%%%%%%%%%%%%%%%%%%%%%%%%%%%




% define\e@alloc as in latex (the existing macros in etex.src
% hard to extend to further register types as they assume specific
% 26x and 27x count range. For compatibility the existing register
% allocation is not changed.


\def\e@alloc#1#2#3#4#5#6{%
  \global\advance#3\@ne
  \e@ch@ck{#3}{#4}{#5}#1%
  \allocationnumber#3\relax
  \global#2#6\allocationnumber
  \wlog{\string#6=\string#1\the\allocationnumber}}%
\gdef\e@ch@ck#1#2#3#4{%
  \ifnum#1<#2\else
    \ifnum#1=#2\relax
      #1\@cclvi
      \ifx\count#4\advance#1 10 \fi
    \fi
    \ifnum#1<#3\relax
    \else
      \errmessage{No room for a new #4}%
    \fi
  \fi}%


%%%%%%
\long\def \@gobble #1{}
\long\def\@firstofone#1{#1}
\def\makeatother{\catcode`\@12\relax}
\input ltluatex.sty


% fix up allocations not to clash with etex.src
% \count registers 256-259 and 267-269 are not (yet) used
% 

% ltluatex uses 258 and 259
% also uses 260 and 261 but change them to 267 and 268
\def\newluafunction{%
  \e@alloc\luafunction\e@alloc@chardef
    {\count267}\m@ne\e@alloc@top
}
\count267=\z@
\def\newwhatsit#1{%
  \e@alloc\whatsit\e@alloc@chardef
    {\count268}\m@ne\e@alloc@top#1%
}
\count268=\z@

\directlua{
local function new_whatsit(name)
  tex_setcount("global", 267, tex_count[267] + 1)
  if tex_count[267] > 65534 then
    latex_error("No room for a new custom whatsit")
    return -1
  end
  texio_write_nl("Custom whatsit " .. name .. " = " .. tex_count[267])
  return tex_count[267]
end
latex.new_whatsit = new_whatsit
}

\catcode`\@=\etatcatcode\relax
%
    \else
      \RequirePackage{ltluatex}
    \fi
  \fi
\fi
%    \end{macrocode}
%
% Simple require wrapper so that |luaotfload| does get loaded.
%    \begin{macrocode}
\def\RequireLuaModule#1{\directlua{require("#1")}}
%    \end{macrocode}
%
% Extended catcode table support.  Use the names from the previous
% \textsf{luatexbase} and \textsf{luatex} packages.
%    \begin{macrocode}
\let\CatcodeTableIniTeX\catcodetable@initex
\let\CatcodeTableString\catcodetable@string
\let\CatcodeTableLaTeX\catcodetable@latex
\let\CatcodeTableLaTeXAtLetter\catcodetable@atletter
%    \end{macrocode}
%
% Additional tables declared in the previous interface.
%    \begin{macrocode}
\newcatcodetable\CatcodeTableOther
\setcatcodetable\CatcodeTableOther{%
  \catcodetable\CatcodeTableString
  \catcode32 12 }
%    \end{macrocode}
%
%    \begin{macrocode}
\newcatcodetable\CatcodeTableExpl
\setcatcodetable\CatcodeTableExpl{%
  \catcodetable\CatcodeTableLaTeX
  \catcode126 10 % tilde is a space char
  \catcode32  9  % space is ignored
  \catcode9   9  % tab also ignored
  \catcode95  11 % underscore letter
  \catcode58  11 % colon letter
}
%    \end{macrocode}
%
% Top level access to catcodetable stack.
%    \begin{macrocode}
\def\BeginCatcodeRegime#1{%
  \@pushcatcodetable
  \catcodetable#1\relax}
%    \end{macrocode}
%
%    \begin{macrocode}
\def\EndCatcodeRegime{%
  \@popcatcodetable}
%    \end{macrocode}
%
% The implementation of the stack is completely
% different, but usage should match.
%    \begin{macrocode}
\let\PushCatcodeTableNumStack\@pushcatcodetable
\let\PopCatcodeTableNumStack\@popcatcodetable
%    \end{macrocode}
%
% \subsubsection{Additional lua code}
%    \begin{macrocode}
\directlua{
%    \end{macrocode}
%
% Remove all registered callbacks, then disable.
% Set to false if optional second argument is |true|.
%    \begin{macrocode}
function luatexbase.reset_callback(name,make_false)
  for _,v in pairs(luatexbase.callback_descriptions(name))
  do
    luatexbase.remove_from_callback(name,v)
  end
  if make_false == true then
    luatexbase.disable_callback(name)
  end
end
%    \end{macrocode}
%
% Emulate the catcodetables table.
% Explicitly fill the table rather than rely on the metatable call to
% |registernumber| as that is unreliable on old Lua\TeX{}.
%    \begin{macrocode}
luatexbase.catcodetables=setmetatable(
 {['latex-package'] = \number\catcodetable@atletter,
  ini    = \number\CatcodeTableIniTeX,
  string = \number\CatcodeTableString,
  other  = \number\CatcodeTableOther,
  latex  = \number\catcodetable@latex,
  expl   = \number\CatcodeTableExpl,
  expl3  = \number\CatcodeTableExpl},
 { __index = function(t,key)
    return luatexbase.registernumber(key) or nil
  end}
)}
%    \end{macrocode}
%
% On old Lua\TeX{} workaround hashtable issues.
% Allocate in \TeX{}, and also directly add to |luatexbase.catcodetables|.
%    \begin{macrocode}
\ifnum\luatexversion<80 %
\def\newcatcodetable#1{%
  \e@alloc\catcodetable\chardef
    \e@alloc@ccodetable@count\m@ne{"8000}#1%
  \initcatcodetable\allocationnumber
  {\escapechar=\m@ne
  \directlua{luatexbase.catcodetables['\string#1']=%
    \the\allocationnumber}}%
}
\fi
%    \end{macrocode}
%
%    \begin{macrocode}
\directlua{
%    \end{macrocode}
%
% |priority_in_callback| returns position in the callback list.
% Not provided by default in |ltluatex| as usually it is just used 
% as a boolean test, for which |in_callback| is provided.
%    \begin{macrocode}
function luatexbase.priority_in_callback (name,description)
  for i,v in ipairs(luatexbase.callback_descriptions(name))
  do
    if v == description then
      return i
    end
  end
  return false
end
%    \end{macrocode}
% 
% The (unreleased) version 0.7 of \textsf{luatexbase} provided this
% boolean test under a different name, so we provide an alias here.
%    \begin{macrocode}
luatexbase.is_active_callback = luatexbase.in_callback
%    \end{macrocode}
%
% |ltluatex| implementation of |provides_module| does not return
% print functions so define modified version here.
%    \begin{macrocode}
luatexbase.base_provides_module=luatexbase.provides_module
function luatexbase.provides_module(info)
  luatexbase.base_provides_module(info)
  return
    function(s) return luatexbase.module_error(info.name, s) end,
    function(s) return luatexbase.module_warning(info.name, s) end,
    function(s) return luatexbase.module_info(info.name, s) end,
    function(s) return luatexbase.module_info(info.name, s) end
end
}
%    \end{macrocode}
%
% Same for attribute table as catcode tables. In old Lua\TeX{}, add to the
% |luatexbase.attributes| table directly.
%    \begin{macrocode}
\ifnum\luatexversion<80 %
\def\attribute#1{%
  \e@alloc\attribute\attributedef
    \e@alloc@attribute@count\m@ne\e@alloc@top#1%
  {\escapechar=\m@ne
  \directlua{luatexbase.attributes['\string#1']=%
    \the\allocationnumber}}%
}
\fi
%    \end{macrocode}
%
% Resolve name clashes and prefixed name issues.
%    \begin{macrocode}
\let\newluatexattribute\newattribute
\let\setluatexattribute\setattribute
\let\unsetluatexattribute\unsetattribute
\let\newluatexcatcodetable\newcatcodetable
\let\setluatexcatcodetable\setcatcodetable
\let\luatexbase@directlua\directlua
\let\luatebase@ensure@primitive\@gobble
%    \end{macrocode}
%
% reset catcode of |@|.
%    \begin{macrocode}
\catcode`\@=\emuatcatcode\relax
%    \end{macrocode}
%
%    \begin{macrocode}
%</emu>
%    \end{macrocode}
%
%    \begin{macrocode}
%<*emu-cmp,emu-mod,emu-loa,emu-reg,emu-att,emu-cct,emu-mcb>
%    \end{macrocode}
%
%    \begin{macrocode}
\ifx\RequirePackage\undefined
  % \iffalse meta-comment
%
% Copyright 2015
% The LaTeX3 Project and any individual authors listed elsewhere
% in this file.
%
% It may be distributed and/or modified under the conditions of
% the LaTeX Project Public License (LPPL), either version 1.3c of
% this license or (at your option) any later version.  The latest
% version of this license is in the file:
%
%   http://www.latex-project.org/lppl.txt
%
%
%
%<emu>\ifx\BeginCatcodeRegime\undefined\else\expandafter\endinput\fi
%<plain>\ifx\newluafunction\undefined\else\expandafter\endinput\fi
%<tex,sty>\ifx
%<sty>  \ProvidesPackage\undefined\begingroup\def\ProvidesPackage
%<tex>  \ProvidesFile\undefined\begingroup\def\ProvidesFile
%<tex,sty>  #1#2[#3]{\endgroup\immediate\write16{Package: #1 #3}}
%<tex,sty>\fi
%<plain>\ProvidesFile{ltluatex.tex}
%<emu>\ProvidesPackage{luatexbase}
%<emu-cmp>\ProvidesPackage{luatexbase-compat}
%<emu-mod>\ProvidesPackage{luatexbase-modutils}
%<emu-loa>\ProvidesPackage{luatexbase-loader}
%<emu-reg>\ProvidesPackage{luatexbase-regs}
%<emu-att>\ProvidesPackage{luatexbase-attr}
%<emu-cct>\ProvidesPackage{luatexbase-cctb}
%<emu-mcb>\ProvidesPackage{luatexbase-mcb}
%<*driver>
\ProvidesFile{luatexbase.dtx}
%</driver>
%<*tex,sty>
[2015/08/01 v1.0a
%</tex,sty>
%<plain>  luatex support for plain TeX (core + catcode stack)
%<emu>  luatexbase interface to luatex
%<emu-cmp>  luatexbase interface to luatex (legacy subpackage)
%<emu-mod>  luatexbase interface to luatex (legacy subpackage)
%<emu-loa>  luatexbase interface to luatex (legacy subpackage)
%<emu-reg>  luatexbase interface to luatex (legacy subpackage)
%<emu-cct>  luatexbase interface to luatex (legacy subpackage)
%<emu-mcb>  luatexbase interface to luatex (legacy subpackage)
%<*tex,sty>
]
%</tex,sty>
%<*driver>
\documentclass{ltxdoc}
\GetFileInfo{luatexbase.dtx}
\begin{document}
\title{\filename\\(Lua\TeX{}-specific support, luatexbase interface)}
\author{David Carlisle and Joseph Wright\footnote{Significant portions
  of the code here are adapted/simplified from the packages \textsf{luatex} and
  \textsf{luatexbase} written by Heiko Oberdiek, \'{E}lie Roux,
  Manuel P\'{e}gouri\'{e}-Gonnar and Philipp Gesang.}}
\date{\filedate}
\maketitle
\setcounter{tocdepth}{2}
\tableofcontents
\DocInput{\filename}
\end{document}
%</driver>
% \fi
%
% \CheckSum{0}
%
% \section{Overview}
%
% Lua\TeX{} adds a number of engine-specific functions to \TeX{}. Several of
% these require set up that is best done in the kernel or need related support
% functions. This file provides \emph{basic} support for Lua\TeX{} at the
% \LaTeXe{} kernel level plus as a loadable package which can be used with plain
% \TeX{} and \LaTeX{}. For \LaTeX{}, the package adds a small number of commands
% which do not fit into the scope of the kernel.
%
% This file contains code for both \TeX{} (to be stored as part of the format)
% and Lua (to be loaded at the start of each job). In the Lua code, the kernel
% uses the namespace |latex| which is \emph{reserved for use by the team}.
%
% The following |\count| registers are used here for register allocation:
% \begin{itemize}
%  \item[\texttt{\string\e@alloc@attribute@count}] Attributes (default 258)
%  \item[\texttt{\string\e@alloc@ccodetable@count}] Category code tables (default 259)
%  \item[\texttt{\string\e@alloc@luafunction@count}] Lua functions (default 260)
%  \item[\texttt{\string\e@alloc@whatsit@count}] User whatsits (default 261)
% \end{itemize}
% (|\count256| is used for |\newmarks| allocation and |\count257|
% is used for\break
% |\newXeTeXintercharclass| with Xe\TeX{}, with code defined in
% \texttt{ltfinal.dtx}).
% With any \LaTeXe{} kernel from 2015 onward these registers are part of
% the block in the extended area reserved by the kernel (prior to 2015 the
% \LaTeXe{} kernel did not provide any functionality for the extended
% allocation area).
%
% \section{Core \TeX{} functionality}
%
% The commands defined here are defined in |ltluatex.def| for
% possible inclusion in a future \LaTeX\ format, however for ease of
% testing they are also currently defined in the package file
% |ltluatex.sty| along with the additional commands described in the
% next section.
%
% \DescribeMacro{\newattribute}
% |\newattribute{|\meta{attribute}|}|\\
% Defines a named \cs{attribute}, indexed from~$1$
% (\emph{i.e.}~|\attribute0| is never defined). Attributes initially
% have the marker value |-"7FFFFFFF| (`unset') set by the engine.
%
% \noindent
% \DescribeMacro{\newcatcodetable}
% |\newcatcodetable{|\meta{catcodetable}|}|\\
% Defines a named \cs{catcodetable}, indexed from~$1$
% (|\catcodetable0| is never assigned). A new catcode table will be
% populated with exactly those values assigned by Ini\TeX{} (as described
% in the Lua\TeX{} manual).
%
% \noindent
% \DescribeMacro{\newluafunction}
% |\newluafunction{|\meta{function}|}|\\
% Defines a named \cs{luafunction}, indexed from~$1$ (Lua indexes from
% $1$ so |\luafunction0| is not available).
%
% \noindent
% \DescribeMacro{\newwhatsit}
% |\newwhatsit{|\meta{whatsit}|}|\\
% Defines a custom \cs{whatsit}, indexed from~$1$.
%
% \noindent
% \DescribeMacro{\catcodetable@initex}
% \DescribeMacro{\catcodetable@string}
% \DescribeMacro{\catcodetable@latex}
% \DescribeMacro{\catcodetable@atletter}
% Predefined category code tables with the obvious assignments. Note
% that the |latex| and |atletter| tables set the full Unicode range
% to the codes predefined by the kernel.
%
% \section{The \texttt{ltluatex} Package \TeX{} functionality}
%
% \noindent
% \DescribeMacro{\setattribute}
% \DescribeMacro{\unsetattribute}
% |\setattribute{|\meta{attribute}|}{|\meta{value}|}|\\
% |\unsetattribute{|\meta{attribute}|}|\\
% Set and unset attributes in a manner analogous to |\setlength|. Note that
% attributes take a marker value when unset so this operation is distinct
% from setting the value to zero.
%
% \noindent
% \DescribeMacro{\setcatcodetable}
% |\setcatcodetable{|\meta{table}|}{|\meta{catcodes}|}|\\
% Sets the \meta{table} (which must have been previously defined) to
% apply the \meta{catcodes} specified.
%
% \noindent
% \DescribeMacro{\setrangecatcode}
% |\setrangecatcode{|\meta{start}|}{|\meta{end}|}{|\meta{catcode}|}|\\
% Sets all characters in the range \meta{start}--\meta{end} inclusive to
% have the \meta{catcode} specified.
%
% \noindent
% \DescribeMacro{\@pushcatcodes}
% \DescribeMacro{\@popcatcodes}
% |\@pushcatcodetable|\\
% |\@popcatcodetable|\\
% This pair of commands enable the current category code r\'{e}gime to
% be saved and restored meaning that arbitrary catcode changes can be made.
% This functionality will normally be used in concert with applying
% catcode tables. For example
% \begin{verbatim}
% \catcode`\Z=4 %
% \@pushcatcodetable
% \catcodetable\catcodetable@latex
% % Code here
% \@popcatcodetable
% \showthe\catcode`\Z
% \end{verbatim}
% will ensure that the `content' is set with normal category codes but
% allow restoration of the non-standard codes at the conclusion. Importantly,
% it does not require that anything is known about the catcode situation in
% advance (\emph{cf.}~a more traditional approach to saving the state of
% targeting characters).
%
% \section{The \texttt{luatexbase} package interface}
%
% The interfaces defined here are closely modelled on the earlier
% \textsf{luatexbase} package, which in turn copies parts of the
% \textsf{luatex} package.
% A closer emulation of the existing \textsf{luatexbase} is provided by the
% package currently called |emu-luatexbase|. Having a different name
% allows both packages to be easily used while testing, although
% possibly |emu-luatexbase| can be the source of an updated
% \textsf{luatexbase} package, depending on how things turn out.
%
% \subsection{\TeX}
% \subsubsection{Alternative catcode table handling}
%
% \noindent
% \DescribeMacro{\CatcodeTableIniTeX}
% \DescribeMacro{\CatcodeTableString}
% \DescribeMacro{\CatcodeTableLaTeX}
% \DescribeMacro{\CatcodeTableLaTeXAtLetter}
% \DescribeMacro{\CatcodeTableOther}
% \DescribeMacro{\CatcodeTableExpl}
% Alternative names for some predefined catcodetables, and some
% additional ones are declared.
%
% \noindent
% \DescribeMacro{\BeginCatcodeRegime}
% \DescribeMacro{\EndCatcodeRegime}
% A simple wrapper around |\@pushcatcodetable| providing aslightly
% different interface.
%
% \noindent
% \DescribeMacro{\PushCatcodeTableNumStack}
% \DescribeMacro{\PopCatcodeTableNumStack}
% These are defined to be aliases for |\@pushcatcodetable| and
% |\@popcatcodetable| although the actual implemenatation is quite different
% to the older packages, the use of the commands should match.
%
% \subsection{Lua}
% \subsubsection{predefined catcode table names}
% As in the previous packages the |catcodetables| array is extended
% with aliases for the predefined tables so you can use |catcodetables.expl| in
% addition to |catcodetables.CatcodeTableExpl|.
%
% \subsubsection{callback queries}
% In the earlier interfaces the function
% |luatexbase.priority_in_callback|
% is provided to return a number indicating th eposition of a
% specified function in a callback list. However it is usually used
% just as a boolean test that the function is registered with the
% callback. |ltluatex| does not expose the priority numbers so this
% function is defined as an alias to the function
% |luatexbase.in_callback| which returns a boolean.
%
% \subsubsection{Module declaration}
% The |luatexbase| version of |provides_module| returns a list of log
% and error functions so that it is usually called as:\\
% |local err, warning, info, log = luatexbase.provides_module({name=..|
%
% \section{Plain \TeX\ interface}
%
% The |ltluatex| interface may be used with plain \TeX\ using 
% |\input ltluatex| this inputs |ltluatex.tex| which inputs |etex.src|
% if it is not already input, and then defines some internal commands to
% allow the \emph{extended} interafce defined by |ltluatex.sty| to be
% defined.
%
% The `luatexbase` package interface may also be used in plain \TeX,
% as before by simply inputting the package, currently |\input emu-luatexbase.sty|.
%
% \section{Lua functionality}
%
% \begingroup
%
% \begingroup\lccode`~=`_
% \lowercase{\endgroup\let~}_
% \catcode`_=12
%
% \section{Allocators in Lua}
%
% \DescribeMacro{luatexbase.new_attribute}
% |luatexbase.new_attribute(|\meta{attribute}|)|\\
% Returns an allocation number for the \meta{attribute}, indexed from~$1$.
% The attribute will be initialised with the marker value |-"7FFFFFFF|
% (`unset'). The attribute allocation sequence is shared with the \TeX{}
% code but this function does \emph{not} define a token using
% |\attributedef|.
% The attribute name is recorded in the |attributes| table. A
% metatable is provided so that the table syntax can be used
% consistently for attributes declared in \TeX\ or lua.
%
% \noindent
% \DescribeMacro{luatexbase.new_whatsit}
% |luatexbase.new_whatsit(|\meta{whatsit}|)|\\
% Returns an allocation number for the custom \meta{whatsit}, indexed from~$1$.
%
% \noindent
% \DescribeMacro{luatexbase.new_bytecode}
% |luatexbase.new_bytecode(|\meta{bytecode}|)|\\
% Returns an allocation number for a bytecode register, indexed from~$1$.
% The optional \meta{name} argument is just used for logging.
%
% \noindent
% \DescribeMacro{luatexbase.new_chunkname}
% |luatexbase.new_chunkname(|\meta{chunkname}|)|\\
% Returns an allocation number for a lua chunk name for use with 
% |\directlua| and |\latelua|, indexed from~$1$.
% The number is returned and also \meta{name} argument is added to the
% |lua.name| array at that index.
%
%
% \section{Lua access to \TeX{} register numbers}
%
% \DescribeMacro{luatexbase.registernumber}
% |luatexbase.registernumer(|\meta{name}|)|\\
% Sometimes (notably in the case of Lua attributes) it is necessary to
% access a register \emph{by number} that has been allocated by \TeX{}.
% This package provides a function to look up the relevant number
% using Lua\TeX{}'s internal tables. So after for example
% |\newattribute\myattrib|, |\myattrib| would be defined by (say)
% |\myattrib=\attribute15|. The function call |luatexbase.registernumer("myattrib")|
% will then return the register number, $15$ in this case. If the string passed
% as argument does not correspond to a token defined by |\attributedef|,
% |\countdef| or similar commands, the Lua value |false| is returned.
%
% As an example, consider the input:
%\begin{verbatim}
% \newcommand\test[1]{%
% \typeout{#1: \expandafter\meaning\csname#1\endcsname^^J
% \space\space\space\space
% \directlua{tex.write(luatexbase.registernumber("#1") or "bad input")}%
% }}
%
% \test{undefinedrubbish}
%
% \test{space}
%
% \test{hbox}
%
% \test{@MM}
%
% \test{@tempdima}
% \test{@tempdimb}
%
% \test{strutbox}
%
% \test{sixt@@n}
%
% \attrbutedef\myattr=12
% \myattr=200
% \test{myattr}
%
%\end{verbatim}
%
% If the demonstration code is processed with Lua\LaTeX{} then the following
% would be produced in the log and terminal output.
%\begin{verbatim}
% undefinedrubbish: \relax
%      bad input
% space: macro:->
%      bad input
% hbox: \hbox
%      bad input
% @MM: \mathchar"4E20
%      20000
% @tempdima: \dimen14
%      14
% @tempdimb: \dimen15
%      15
% strutbox: \char"B
%      11
% sixt@@n: \char"10
%      16
% myattr: \attribute12
%      12
%\end{verbatim}
%
% Notice how undefined commands, or commands unrelated to registers
% do not produce an error, just return |false| and so print
% |bad input| here. Note also that commands defined by |\newbox| work and
% return the number of the box register even though the actual command
% holding this number is a |\chardef| defined token (there is no
% |\boxdef|).
%
% \subsection{Module utilities}
%
% \DescribeMacro{luatexbase.provides_module}
% |luatexbase.provides_module(|\meta{info}|)|\\
% This function is used by modules to identify themselves; the |info| should be
% a table containing information about the module. The required field
% |name| must contain the name of the module. It is recommended to provide a
% field |date| in the usual \LaTeX{} format |yyyy/mm/dd|. Optional fields
% |version| (a string) and |description| may be used if present. This
% information will be recorded in the log. Other fields are ignored.
%
% \noindent
% \DescribeMacro{luatexbase.module_info}
% \DescribeMacro{luatexbase.module_warning}
% \DescribeMacro{luatexbase.module_error}
% |luatexbase.module_info(|\meta{module}, \meta{text}|)|\\
% |luatexbase.module_warning(|\meta{module}, \meta{text}|)|\\
% |luatexbase.module_error(|\meta{module}, \meta{text}|)|\\
% These functions are similar to \LaTeX{}'s |\PackageError|, |\PackageWarning|
% and |\PackageInfo| in the way they format the output.  No automatic line
% breaking is done, you may still use |\n| as usual for that, and the name of
% the package will be prepended to each output line.
%
% Note that |luatexbase.module_error| raises an actual Lua error with |error()|,
% which currently means a call stack will be dumped. While this may not
% look pretty, at least it provides useful information for tracking the
% error down.
%
% \subsection{Callback management}
%
% \noindent
% \DescribeMacro{luatexbase.add_to_callback}
% |luatexbase.add_to_callback(|^^A
% \meta{callback}, \meta{function}, \meta{description}|)|
% Registers the \meta{function} into the \meta{callback} with a textual
% \meta{description} of the function. Functions are inserted into the callback
% in the order loaded.
%
% \noindent
% \DescribeMacro{luatexbase.remove_from_callback}
% |luatexbase.remove_from_callback(|\meta{callback}, \meta{description}|)|
% Removes the function with \meta{description} from the \meta{callback}.
% The removed function and its description 
% are returned as the results of this function.
%
% \noindent
% \DescribeMacro{luatexbase.in_callback}
% |luatexbase.in_callback(|\meta{callback}, \meta{description}|)|
% Checks if the \meta{description} matches one of the functions added
% to the list for the \meta{callback}, returning a boolean value.
%
% \noindent
% \DescribeMacro{luatexbase.disable_callback}
% |luatexbase.disable_callback(|\meta{callback}|)|
% Sets the \meta{callback} to \texttt{false} as described in the Lua\TeX{}
% manual for the underlying \texttt{callback.register} built-in. Callbacks
% will only be set to false (and thus be skipped entirely) if there are
% no functions registered using the callback.
%
% \noindent
% \DescribeMacro{luatexbase.callback_descriptions}
% A list of the descriptions of functions registered to the specified
% callback is returned. |{}| is returned if there are no functions registered.
%
% \noindent
% \DescribeMacro{luatexbase.create_callback}
% |luatexbase.create_callback(|\meta{name},meta{type},\meta{default}|)|
% Defines a user defined callback. The last argument is a default
% functtion of |false|.
%
% \noindent
% \DescribeMacro{luatexbase.call_callback}
% |luatexbase.create_callback(|\meta{name},\ldots|)|
% Calls a user defined callback with the supplied arguments.
%
% \endgroup
%
% \StopEventually{}
%
% \section{Implementation}
%
%
% \subsection{Plain \TeX\ support}
%
%    \begin{macrocode}
%<*plain>
%    \end{macrocode}
%
%    \begin{macrocode}
\edef\etatcatcode{\the\catcode`\@}
\catcode`\@=11
%    \end{macrocode}
%
%    \begin{macrocode}
\ifx\e@alloc\@undefined\else
  \expandafter\endinput
\fi
%    \end{macrocode}
%
% \subsubsection{Fixes to \texttt{etex.src}/\texttt{etex.sty}}
%
% These could and probably should be made directly in an
% update to etex.src which already has some Lua\TeX{}-specific
% code, but does not define the correct range for Lua\TeX{}.
%
%    \begin{macrocode}
% 2015-07-13 higher range in luatex
\edef \et@xmaxregs {\ifx\directlua\@undefined 32768\else 65536\fi}
% luatex/xetex also allow more math fam
\edef \et@xmaxfam {\ifx\Umathchar\@undefined\sixt@@n\else\@cclvi\fi}
%    \end{macrocode}
%
%    \begin{macrocode}
\count 270=\et@xmaxregs % locally allocates \count registers 32767, 32766, ...
\count 271=\et@xmaxregs % ditto for \dimen registers
\count 272=\et@xmaxregs % ditto for \skip registers
\count 273=\et@xmaxregs % ditto for \muskip registers
\count 274=\et@xmaxregs % ditto for \box registers
\count 275=\et@xmaxregs % ditto for \toks registers
\count 276=\et@xmaxregs % ditto for \marks classes
%    \end{macrocode}
%
%    \begin{macrocode}
% and 256 or 16 fam
\outer\def\newfam{\alloc@8\fam\chardef\et@xmaxfam}
%    \end{macrocode}
%
% End of proposed changes to \texttt{etex.src}
%
% \subsubsection{luatex specific settings}
% Switch to global cf |luatex.sty| to leave room for inserts
% not really needed for Lua\TeX{} but possibly most compatible
% with existing use.
%    \begin{macrocode}
\let\newcount\globcount
\let\newdimen\globdimen
\let\newskip\globskip
\let\newbox\globbox
%    \end{macrocode}
%
% Define|\e@alloc| as in latex (the existing macros in |etex.src|
% hard to extend to further register types as they assume specific
% 26x and 27x count range. For compatibility the existing register
% allocation is not changed.
%
%    \begin{macrocode}
\chardef\e@alloc@top=65535
\let\e@alloc@chardef\chardef
%    \end{macrocode}
%
%    \begin{macrocode}
\def\e@alloc#1#2#3#4#5#6{%
  \global\advance#3\@ne
  \e@ch@ck{#3}{#4}{#5}#1%
  \allocationnumber#3\relax
  \global#2#6\allocationnumber
  \wlog{\string#6=\string#1\the\allocationnumber}}%
%    \end{macrocode}
%
%    \begin{macrocode}
\gdef\e@ch@ck#1#2#3#4{%
  \ifnum#1<#2\else
    \ifnum#1=#2\relax
      #1\@cclvi
      \ifx\count#4\advance#1 10 \fi
    \fi
    \ifnum#1<#3\relax
    \else
      \errmessage{No room for a new \string#4}%
    \fi
  \fi}%
%    \end{macrocode}
%
% Two simple \LaTeX\ macros used in |ltlatex.sty|.
%    \begin{macrocode}
\long\def \@gobble #1{}
\long\def\@firstofone#1{#1}
%    \end{macrocode}
%
%    \begin{macrocode}
% Fix up allocations not to clash with |etex.src|.
%    \end{macrocode}
%
%    \begin{macrocode}
\newcount\e@alloc@attribute@count
\newcount\e@alloc@ccodetable@count
\newcount\e@alloc@luafunction@count
\newcount\e@alloc@whatsit@count
%    \end{macrocode}
%
%    \begin{macrocode}
% Input the main code.
\input ltluatex.sty\relax
%    \end{macrocode}
%
% Reset the catcode of |@|.
%    \begin{macrocode}
\catcode`\@=\etatcatcode\relax
%    \end{macrocode}
%
%    \begin{macrocode}
%</plain>
%    \end{macrocode}
%
% \subsection{luatexbase interface}
%
%    \begin{macrocode}
%<*emu>
\edef\emuatcatcode{\the\catcode`\@}
\catcode`\@=11
%    \end{macrocode}
%
%    \begin{macrocode}
% emu-luatexbase.sty v 0.01 2015/07/14
% minimal luatexbase emulation, for testing with
% current luaotfload package
%    \end{macrocode}
%
%
%    \begin{macrocode}
\ifx\e@alloc\@undefined
%    \end{macrocode}
%
%    \begin{macrocode}
% In pre 2014 latex, or plain TeX load |etex.{sty,src}|
%    \end{macrocode}
%
%    \begin{macrocode}
  \ifx\documentclass\@undefined
    \ifx\loccount\@undefined
      \input{etex.src}%
    \fi
    \catcode`\@=11 %
  \else
    \RequirePackage{etex}
  \fi
%    \end{macrocode}
% Load the code for the plain interface (which also works in old
% \LaTeX).
%    \begin{macrocode}
  
\edef\etatcatcode{\the\catcode`\@}
\catcode`\@=11

\ifx\e@alloc\@undefined\else
\expandafter\endinput
\fi

%
% fixes to etex.src, 
% These couldand probably should be made directly in an
% update to etex.src which already has some luatex-specific
% code, but does not define the correct range for luatex.

% 2015-07-13 higher range in luatex
\edef \et@xmaxregs {\ifx\directlua\@undefined 32768\else 65536\fi}

\count 270=\et@xmaxregs % locally allocates \count registers 32767, 32766, ...
\count 271=\et@xmaxregs % ditto for \dimen registers
\count 272=\et@xmaxregs % ditto for \skip registers
\count 273=\et@xmaxregs % ditto for \muskip registers
\count 274=\et@xmaxregs % ditto for \box registers
\count 275=\et@xmaxregs % ditto for \toks registers
\count 276=\et@xmaxregs % ditto for \marks classes

%%%%%%%%%%%%%%%%%%%%%%%%%%%%%%%%%%%%%%%%%%%%%%%%%%%%%%




% define\e@alloc as in latex (the existing macros in etex.src
% hard to extend to further register types as they assume specific
% 26x and 27x count range. For compatibility the existing register
% allocation is not changed.


\def\e@alloc#1#2#3#4#5#6{%
  \global\advance#3\@ne
  \e@ch@ck{#3}{#4}{#5}#1%
  \allocationnumber#3\relax
  \global#2#6\allocationnumber
  \wlog{\string#6=\string#1\the\allocationnumber}}%
\gdef\e@ch@ck#1#2#3#4{%
  \ifnum#1<#2\else
    \ifnum#1=#2\relax
      #1\@cclvi
      \ifx\count#4\advance#1 10 \fi
    \fi
    \ifnum#1<#3\relax
    \else
      \errmessage{No room for a new #4}%
    \fi
  \fi}%


%%%%%%
\long\def \@gobble #1{}
\long\def\@firstofone#1{#1}
\def\makeatother{\catcode`\@12\relax}
\input ltluatex.sty


% fix up allocations not to clash with etex.src
% \count registers 256-259 and 267-269 are not (yet) used
% 

% ltluatex uses 258 and 259
% also uses 260 and 261 but change them to 267 and 268
\def\newluafunction{%
  \e@alloc\luafunction\e@alloc@chardef
    {\count267}\m@ne\e@alloc@top
}
\count267=\z@
\def\newwhatsit#1{%
  \e@alloc\whatsit\e@alloc@chardef
    {\count268}\m@ne\e@alloc@top#1%
}
\count268=\z@

\directlua{
local function new_whatsit(name)
  tex_setcount("global", 267, tex_count[267] + 1)
  if tex_count[267] > 65534 then
    latex_error("No room for a new custom whatsit")
    return -1
  end
  texio_write_nl("Custom whatsit " .. name .. " = " .. tex_count[267])
  return tex_count[267]
end
latex.new_whatsit = new_whatsit
}

\catcode`\@=\etatcatcode\relax
%
\else
%    \end{macrocode}
% In \LaTeX\ releases from 2015/01/01 load |ltluatex|,
% load |ltluatex.sty| unless things already defined.
%    \begin{macrocode}
  \ifx\catcodetable@initex\@undefined
    \ifx\RequirePackage\@undefined
      
\edef\etatcatcode{\the\catcode`\@}
\catcode`\@=11

\ifx\e@alloc\@undefined\else
\expandafter\endinput
\fi

%
% fixes to etex.src, 
% These couldand probably should be made directly in an
% update to etex.src which already has some luatex-specific
% code, but does not define the correct range for luatex.

% 2015-07-13 higher range in luatex
\edef \et@xmaxregs {\ifx\directlua\@undefined 32768\else 65536\fi}

\count 270=\et@xmaxregs % locally allocates \count registers 32767, 32766, ...
\count 271=\et@xmaxregs % ditto for \dimen registers
\count 272=\et@xmaxregs % ditto for \skip registers
\count 273=\et@xmaxregs % ditto for \muskip registers
\count 274=\et@xmaxregs % ditto for \box registers
\count 275=\et@xmaxregs % ditto for \toks registers
\count 276=\et@xmaxregs % ditto for \marks classes

%%%%%%%%%%%%%%%%%%%%%%%%%%%%%%%%%%%%%%%%%%%%%%%%%%%%%%




% define\e@alloc as in latex (the existing macros in etex.src
% hard to extend to further register types as they assume specific
% 26x and 27x count range. For compatibility the existing register
% allocation is not changed.


\def\e@alloc#1#2#3#4#5#6{%
  \global\advance#3\@ne
  \e@ch@ck{#3}{#4}{#5}#1%
  \allocationnumber#3\relax
  \global#2#6\allocationnumber
  \wlog{\string#6=\string#1\the\allocationnumber}}%
\gdef\e@ch@ck#1#2#3#4{%
  \ifnum#1<#2\else
    \ifnum#1=#2\relax
      #1\@cclvi
      \ifx\count#4\advance#1 10 \fi
    \fi
    \ifnum#1<#3\relax
    \else
      \errmessage{No room for a new #4}%
    \fi
  \fi}%


%%%%%%
\long\def \@gobble #1{}
\long\def\@firstofone#1{#1}
\def\makeatother{\catcode`\@12\relax}
\input ltluatex.sty


% fix up allocations not to clash with etex.src
% \count registers 256-259 and 267-269 are not (yet) used
% 

% ltluatex uses 258 and 259
% also uses 260 and 261 but change them to 267 and 268
\def\newluafunction{%
  \e@alloc\luafunction\e@alloc@chardef
    {\count267}\m@ne\e@alloc@top
}
\count267=\z@
\def\newwhatsit#1{%
  \e@alloc\whatsit\e@alloc@chardef
    {\count268}\m@ne\e@alloc@top#1%
}
\count268=\z@

\directlua{
local function new_whatsit(name)
  tex_setcount("global", 267, tex_count[267] + 1)
  if tex_count[267] > 65534 then
    latex_error("No room for a new custom whatsit")
    return -1
  end
  texio_write_nl("Custom whatsit " .. name .. " = " .. tex_count[267])
  return tex_count[267]
end
latex.new_whatsit = new_whatsit
}

\catcode`\@=\etatcatcode\relax
%
    \else
      \RequirePackage{ltluatex}
    \fi
  \fi
\fi
%    \end{macrocode}
%
% Simple require wrapper so that |luaotfload| does get loaded.
%    \begin{macrocode}
\def\RequireLuaModule#1{\directlua{require("#1")}}
%    \end{macrocode}
%
% Extended catcode table support.  Use the names from the previous
% \textsf{luatexbase} and \textsf{luatex} packages.
%    \begin{macrocode}
\let\CatcodeTableIniTeX\catcodetable@initex
\let\CatcodeTableString\catcodetable@string
\let\CatcodeTableLaTeX\catcodetable@latex
\let\CatcodeTableLaTeXAtLetter\catcodetable@atletter
%    \end{macrocode}
%
% Additional tables declared in the previous interface.
%    \begin{macrocode}
\newcatcodetable\CatcodeTableOther
\setcatcodetable\CatcodeTableOther{%
  \catcodetable\CatcodeTableString
  \catcode32 12 }
%    \end{macrocode}
%
%    \begin{macrocode}
\newcatcodetable\CatcodeTableExpl
\setcatcodetable\CatcodeTableExpl{%
  \catcodetable\CatcodeTableLaTeX
  \catcode126 10 % tilde is a space char
  \catcode32  9  % space is ignored
  \catcode9   9  % tab also ignored
  \catcode95  11 % underscore letter
  \catcode58  11 % colon letter
}
%    \end{macrocode}
%
% Top level access to catcodetable stack.
%    \begin{macrocode}
\def\BeginCatcodeRegime#1{%
  \@pushcatcodetable
  \catcodetable#1\relax}
%    \end{macrocode}
%
%    \begin{macrocode}
\def\EndCatcodeRegime{%
  \@popcatcodetable}
%    \end{macrocode}
%
% The implementation of the stack is completely
% different, but usage should match.
%    \begin{macrocode}
\let\PushCatcodeTableNumStack\@pushcatcodetable
\let\PopCatcodeTableNumStack\@popcatcodetable
%    \end{macrocode}
%
% \subsubsection{Additional lua code}
%    \begin{macrocode}
\directlua{
%    \end{macrocode}
%
% Remove all registered callbacks, then disable.
% Set to false if optional second argument is |true|.
%    \begin{macrocode}
function luatexbase.reset_callback(name,make_false)
  for _,v in pairs(luatexbase.callback_descriptions(name))
  do
    luatexbase.remove_from_callback(name,v)
  end
  if make_false == true then
    luatexbase.disable_callback(name)
  end
end
%    \end{macrocode}
%
% Emulate the catcodetables table.
% Explicitly fill the table rather than rely on the metatable call to
% |registernumber| as that is unreliable on old Lua\TeX{}.
%    \begin{macrocode}
luatexbase.catcodetables=setmetatable(
 {['latex-package'] = \number\catcodetable@atletter,
  ini    = \number\CatcodeTableIniTeX,
  string = \number\CatcodeTableString,
  other  = \number\CatcodeTableOther,
  latex  = \number\catcodetable@latex,
  expl   = \number\CatcodeTableExpl,
  expl3  = \number\CatcodeTableExpl},
 { __index = function(t,key)
    return luatexbase.registernumber(key) or nil
  end}
)}
%    \end{macrocode}
%
% On old Lua\TeX{} workaround hashtable issues.
% Allocate in \TeX{}, and also directly add to |luatexbase.catcodetables|.
%    \begin{macrocode}
\ifnum\luatexversion<80 %
\def\newcatcodetable#1{%
  \e@alloc\catcodetable\chardef
    \e@alloc@ccodetable@count\m@ne{"8000}#1%
  \initcatcodetable\allocationnumber
  {\escapechar=\m@ne
  \directlua{luatexbase.catcodetables['\string#1']=%
    \the\allocationnumber}}%
}
\fi
%    \end{macrocode}
%
%    \begin{macrocode}
\directlua{
%    \end{macrocode}
%
% |priority_in_callback| returns position in the callback list.
% Not provided by default in |ltluatex| as usually it is just used 
% as a boolean test, for which |in_callback| is provided.
%    \begin{macrocode}
function luatexbase.priority_in_callback (name,description)
  for i,v in ipairs(luatexbase.callback_descriptions(name))
  do
    if v == description then
      return i
    end
  end
  return false
end
%    \end{macrocode}
% 
% The (unreleased) version 0.7 of \textsf{luatexbase} provided this
% boolean test under a different name, so we provide an alias here.
%    \begin{macrocode}
luatexbase.is_active_callback = luatexbase.in_callback
%    \end{macrocode}
%
% |ltluatex| implementation of |provides_module| does not return
% print functions so define modified version here.
%    \begin{macrocode}
luatexbase.base_provides_module=luatexbase.provides_module
function luatexbase.provides_module(info)
  luatexbase.base_provides_module(info)
  return
    function(s) return luatexbase.module_error(info.name, s) end,
    function(s) return luatexbase.module_warning(info.name, s) end,
    function(s) return luatexbase.module_info(info.name, s) end,
    function(s) return luatexbase.module_info(info.name, s) end
end
}
%    \end{macrocode}
%
% Same for attribute table as catcode tables. In old Lua\TeX{}, add to the
% |luatexbase.attributes| table directly.
%    \begin{macrocode}
\ifnum\luatexversion<80 %
\def\attribute#1{%
  \e@alloc\attribute\attributedef
    \e@alloc@attribute@count\m@ne\e@alloc@top#1%
  {\escapechar=\m@ne
  \directlua{luatexbase.attributes['\string#1']=%
    \the\allocationnumber}}%
}
\fi
%    \end{macrocode}
%
% Resolve name clashes and prefixed name issues.
%    \begin{macrocode}
\let\newluatexattribute\newattribute
\let\setluatexattribute\setattribute
\let\unsetluatexattribute\unsetattribute
\let\newluatexcatcodetable\newcatcodetable
\let\setluatexcatcodetable\setcatcodetable
\let\luatexbase@directlua\directlua
\let\luatebase@ensure@primitive\@gobble
%    \end{macrocode}
%
% reset catcode of |@|.
%    \begin{macrocode}
\catcode`\@=\emuatcatcode\relax
%    \end{macrocode}
%
%    \begin{macrocode}
%</emu>
%    \end{macrocode}
%
%    \begin{macrocode}
%<*emu-cmp,emu-mod,emu-loa,emu-reg,emu-att,emu-cct,emu-mcb>
%    \end{macrocode}
%
%    \begin{macrocode}
\ifx\RequirePackage\undefined
  % \iffalse meta-comment
%
% Copyright 2015
% The LaTeX3 Project and any individual authors listed elsewhere
% in this file.
%
% It may be distributed and/or modified under the conditions of
% the LaTeX Project Public License (LPPL), either version 1.3c of
% this license or (at your option) any later version.  The latest
% version of this license is in the file:
%
%   http://www.latex-project.org/lppl.txt
%
%
%
%<emu>\ifx\BeginCatcodeRegime\undefined\else\expandafter\endinput\fi
%<plain>\ifx\newluafunction\undefined\else\expandafter\endinput\fi
%<tex,sty>\ifx
%<sty>  \ProvidesPackage\undefined\begingroup\def\ProvidesPackage
%<tex>  \ProvidesFile\undefined\begingroup\def\ProvidesFile
%<tex,sty>  #1#2[#3]{\endgroup\immediate\write16{Package: #1 #3}}
%<tex,sty>\fi
%<plain>\ProvidesFile{ltluatex.tex}
%<emu>\ProvidesPackage{luatexbase}
%<emu-cmp>\ProvidesPackage{luatexbase-compat}
%<emu-mod>\ProvidesPackage{luatexbase-modutils}
%<emu-loa>\ProvidesPackage{luatexbase-loader}
%<emu-reg>\ProvidesPackage{luatexbase-regs}
%<emu-att>\ProvidesPackage{luatexbase-attr}
%<emu-cct>\ProvidesPackage{luatexbase-cctb}
%<emu-mcb>\ProvidesPackage{luatexbase-mcb}
%<*driver>
\ProvidesFile{luatexbase.dtx}
%</driver>
%<*tex,sty>
[2015/08/01 v1.0a
%</tex,sty>
%<plain>  luatex support for plain TeX (core + catcode stack)
%<emu>  luatexbase interface to luatex
%<emu-cmp>  luatexbase interface to luatex (legacy subpackage)
%<emu-mod>  luatexbase interface to luatex (legacy subpackage)
%<emu-loa>  luatexbase interface to luatex (legacy subpackage)
%<emu-reg>  luatexbase interface to luatex (legacy subpackage)
%<emu-cct>  luatexbase interface to luatex (legacy subpackage)
%<emu-mcb>  luatexbase interface to luatex (legacy subpackage)
%<*tex,sty>
]
%</tex,sty>
%<*driver>
\documentclass{ltxdoc}
\GetFileInfo{luatexbase.dtx}
\begin{document}
\title{\filename\\(Lua\TeX{}-specific support, luatexbase interface)}
\author{David Carlisle and Joseph Wright\footnote{Significant portions
  of the code here are adapted/simplified from the packages \textsf{luatex} and
  \textsf{luatexbase} written by Heiko Oberdiek, \'{E}lie Roux,
  Manuel P\'{e}gouri\'{e}-Gonnar and Philipp Gesang.}}
\date{\filedate}
\maketitle
\setcounter{tocdepth}{2}
\tableofcontents
\DocInput{\filename}
\end{document}
%</driver>
% \fi
%
% \CheckSum{0}
%
% \section{Overview}
%
% Lua\TeX{} adds a number of engine-specific functions to \TeX{}. Several of
% these require set up that is best done in the kernel or need related support
% functions. This file provides \emph{basic} support for Lua\TeX{} at the
% \LaTeXe{} kernel level plus as a loadable package which can be used with plain
% \TeX{} and \LaTeX{}. For \LaTeX{}, the package adds a small number of commands
% which do not fit into the scope of the kernel.
%
% This file contains code for both \TeX{} (to be stored as part of the format)
% and Lua (to be loaded at the start of each job). In the Lua code, the kernel
% uses the namespace |latex| which is \emph{reserved for use by the team}.
%
% The following |\count| registers are used here for register allocation:
% \begin{itemize}
%  \item[\texttt{\string\e@alloc@attribute@count}] Attributes (default 258)
%  \item[\texttt{\string\e@alloc@ccodetable@count}] Category code tables (default 259)
%  \item[\texttt{\string\e@alloc@luafunction@count}] Lua functions (default 260)
%  \item[\texttt{\string\e@alloc@whatsit@count}] User whatsits (default 261)
% \end{itemize}
% (|\count256| is used for |\newmarks| allocation and |\count257|
% is used for\break
% |\newXeTeXintercharclass| with Xe\TeX{}, with code defined in
% \texttt{ltfinal.dtx}).
% With any \LaTeXe{} kernel from 2015 onward these registers are part of
% the block in the extended area reserved by the kernel (prior to 2015 the
% \LaTeXe{} kernel did not provide any functionality for the extended
% allocation area).
%
% \section{Core \TeX{} functionality}
%
% The commands defined here are defined in |ltluatex.def| for
% possible inclusion in a future \LaTeX\ format, however for ease of
% testing they are also currently defined in the package file
% |ltluatex.sty| along with the additional commands described in the
% next section.
%
% \DescribeMacro{\newattribute}
% |\newattribute{|\meta{attribute}|}|\\
% Defines a named \cs{attribute}, indexed from~$1$
% (\emph{i.e.}~|\attribute0| is never defined). Attributes initially
% have the marker value |-"7FFFFFFF| (`unset') set by the engine.
%
% \noindent
% \DescribeMacro{\newcatcodetable}
% |\newcatcodetable{|\meta{catcodetable}|}|\\
% Defines a named \cs{catcodetable}, indexed from~$1$
% (|\catcodetable0| is never assigned). A new catcode table will be
% populated with exactly those values assigned by Ini\TeX{} (as described
% in the Lua\TeX{} manual).
%
% \noindent
% \DescribeMacro{\newluafunction}
% |\newluafunction{|\meta{function}|}|\\
% Defines a named \cs{luafunction}, indexed from~$1$ (Lua indexes from
% $1$ so |\luafunction0| is not available).
%
% \noindent
% \DescribeMacro{\newwhatsit}
% |\newwhatsit{|\meta{whatsit}|}|\\
% Defines a custom \cs{whatsit}, indexed from~$1$.
%
% \noindent
% \DescribeMacro{\catcodetable@initex}
% \DescribeMacro{\catcodetable@string}
% \DescribeMacro{\catcodetable@latex}
% \DescribeMacro{\catcodetable@atletter}
% Predefined category code tables with the obvious assignments. Note
% that the |latex| and |atletter| tables set the full Unicode range
% to the codes predefined by the kernel.
%
% \section{The \texttt{ltluatex} Package \TeX{} functionality}
%
% \noindent
% \DescribeMacro{\setattribute}
% \DescribeMacro{\unsetattribute}
% |\setattribute{|\meta{attribute}|}{|\meta{value}|}|\\
% |\unsetattribute{|\meta{attribute}|}|\\
% Set and unset attributes in a manner analogous to |\setlength|. Note that
% attributes take a marker value when unset so this operation is distinct
% from setting the value to zero.
%
% \noindent
% \DescribeMacro{\setcatcodetable}
% |\setcatcodetable{|\meta{table}|}{|\meta{catcodes}|}|\\
% Sets the \meta{table} (which must have been previously defined) to
% apply the \meta{catcodes} specified.
%
% \noindent
% \DescribeMacro{\setrangecatcode}
% |\setrangecatcode{|\meta{start}|}{|\meta{end}|}{|\meta{catcode}|}|\\
% Sets all characters in the range \meta{start}--\meta{end} inclusive to
% have the \meta{catcode} specified.
%
% \noindent
% \DescribeMacro{\@pushcatcodes}
% \DescribeMacro{\@popcatcodes}
% |\@pushcatcodetable|\\
% |\@popcatcodetable|\\
% This pair of commands enable the current category code r\'{e}gime to
% be saved and restored meaning that arbitrary catcode changes can be made.
% This functionality will normally be used in concert with applying
% catcode tables. For example
% \begin{verbatim}
% \catcode`\Z=4 %
% \@pushcatcodetable
% \catcodetable\catcodetable@latex
% % Code here
% \@popcatcodetable
% \showthe\catcode`\Z
% \end{verbatim}
% will ensure that the `content' is set with normal category codes but
% allow restoration of the non-standard codes at the conclusion. Importantly,
% it does not require that anything is known about the catcode situation in
% advance (\emph{cf.}~a more traditional approach to saving the state of
% targeting characters).
%
% \section{The \texttt{luatexbase} package interface}
%
% The interfaces defined here are closely modelled on the earlier
% \textsf{luatexbase} package, which in turn copies parts of the
% \textsf{luatex} package.
% A closer emulation of the existing \textsf{luatexbase} is provided by the
% package currently called |emu-luatexbase|. Having a different name
% allows both packages to be easily used while testing, although
% possibly |emu-luatexbase| can be the source of an updated
% \textsf{luatexbase} package, depending on how things turn out.
%
% \subsection{\TeX}
% \subsubsection{Alternative catcode table handling}
%
% \noindent
% \DescribeMacro{\CatcodeTableIniTeX}
% \DescribeMacro{\CatcodeTableString}
% \DescribeMacro{\CatcodeTableLaTeX}
% \DescribeMacro{\CatcodeTableLaTeXAtLetter}
% \DescribeMacro{\CatcodeTableOther}
% \DescribeMacro{\CatcodeTableExpl}
% Alternative names for some predefined catcodetables, and some
% additional ones are declared.
%
% \noindent
% \DescribeMacro{\BeginCatcodeRegime}
% \DescribeMacro{\EndCatcodeRegime}
% A simple wrapper around |\@pushcatcodetable| providing aslightly
% different interface.
%
% \noindent
% \DescribeMacro{\PushCatcodeTableNumStack}
% \DescribeMacro{\PopCatcodeTableNumStack}
% These are defined to be aliases for |\@pushcatcodetable| and
% |\@popcatcodetable| although the actual implemenatation is quite different
% to the older packages, the use of the commands should match.
%
% \subsection{Lua}
% \subsubsection{predefined catcode table names}
% As in the previous packages the |catcodetables| array is extended
% with aliases for the predefined tables so you can use |catcodetables.expl| in
% addition to |catcodetables.CatcodeTableExpl|.
%
% \subsubsection{callback queries}
% In the earlier interfaces the function
% |luatexbase.priority_in_callback|
% is provided to return a number indicating th eposition of a
% specified function in a callback list. However it is usually used
% just as a boolean test that the function is registered with the
% callback. |ltluatex| does not expose the priority numbers so this
% function is defined as an alias to the function
% |luatexbase.in_callback| which returns a boolean.
%
% \subsubsection{Module declaration}
% The |luatexbase| version of |provides_module| returns a list of log
% and error functions so that it is usually called as:\\
% |local err, warning, info, log = luatexbase.provides_module({name=..|
%
% \section{Plain \TeX\ interface}
%
% The |ltluatex| interface may be used with plain \TeX\ using 
% |\input ltluatex| this inputs |ltluatex.tex| which inputs |etex.src|
% if it is not already input, and then defines some internal commands to
% allow the \emph{extended} interafce defined by |ltluatex.sty| to be
% defined.
%
% The `luatexbase` package interface may also be used in plain \TeX,
% as before by simply inputting the package, currently |\input emu-luatexbase.sty|.
%
% \section{Lua functionality}
%
% \begingroup
%
% \begingroup\lccode`~=`_
% \lowercase{\endgroup\let~}_
% \catcode`_=12
%
% \section{Allocators in Lua}
%
% \DescribeMacro{luatexbase.new_attribute}
% |luatexbase.new_attribute(|\meta{attribute}|)|\\
% Returns an allocation number for the \meta{attribute}, indexed from~$1$.
% The attribute will be initialised with the marker value |-"7FFFFFFF|
% (`unset'). The attribute allocation sequence is shared with the \TeX{}
% code but this function does \emph{not} define a token using
% |\attributedef|.
% The attribute name is recorded in the |attributes| table. A
% metatable is provided so that the table syntax can be used
% consistently for attributes declared in \TeX\ or lua.
%
% \noindent
% \DescribeMacro{luatexbase.new_whatsit}
% |luatexbase.new_whatsit(|\meta{whatsit}|)|\\
% Returns an allocation number for the custom \meta{whatsit}, indexed from~$1$.
%
% \noindent
% \DescribeMacro{luatexbase.new_bytecode}
% |luatexbase.new_bytecode(|\meta{bytecode}|)|\\
% Returns an allocation number for a bytecode register, indexed from~$1$.
% The optional \meta{name} argument is just used for logging.
%
% \noindent
% \DescribeMacro{luatexbase.new_chunkname}
% |luatexbase.new_chunkname(|\meta{chunkname}|)|\\
% Returns an allocation number for a lua chunk name for use with 
% |\directlua| and |\latelua|, indexed from~$1$.
% The number is returned and also \meta{name} argument is added to the
% |lua.name| array at that index.
%
%
% \section{Lua access to \TeX{} register numbers}
%
% \DescribeMacro{luatexbase.registernumber}
% |luatexbase.registernumer(|\meta{name}|)|\\
% Sometimes (notably in the case of Lua attributes) it is necessary to
% access a register \emph{by number} that has been allocated by \TeX{}.
% This package provides a function to look up the relevant number
% using Lua\TeX{}'s internal tables. So after for example
% |\newattribute\myattrib|, |\myattrib| would be defined by (say)
% |\myattrib=\attribute15|. The function call |luatexbase.registernumer("myattrib")|
% will then return the register number, $15$ in this case. If the string passed
% as argument does not correspond to a token defined by |\attributedef|,
% |\countdef| or similar commands, the Lua value |false| is returned.
%
% As an example, consider the input:
%\begin{verbatim}
% \newcommand\test[1]{%
% \typeout{#1: \expandafter\meaning\csname#1\endcsname^^J
% \space\space\space\space
% \directlua{tex.write(luatexbase.registernumber("#1") or "bad input")}%
% }}
%
% \test{undefinedrubbish}
%
% \test{space}
%
% \test{hbox}
%
% \test{@MM}
%
% \test{@tempdima}
% \test{@tempdimb}
%
% \test{strutbox}
%
% \test{sixt@@n}
%
% \attrbutedef\myattr=12
% \myattr=200
% \test{myattr}
%
%\end{verbatim}
%
% If the demonstration code is processed with Lua\LaTeX{} then the following
% would be produced in the log and terminal output.
%\begin{verbatim}
% undefinedrubbish: \relax
%      bad input
% space: macro:->
%      bad input
% hbox: \hbox
%      bad input
% @MM: \mathchar"4E20
%      20000
% @tempdima: \dimen14
%      14
% @tempdimb: \dimen15
%      15
% strutbox: \char"B
%      11
% sixt@@n: \char"10
%      16
% myattr: \attribute12
%      12
%\end{verbatim}
%
% Notice how undefined commands, or commands unrelated to registers
% do not produce an error, just return |false| and so print
% |bad input| here. Note also that commands defined by |\newbox| work and
% return the number of the box register even though the actual command
% holding this number is a |\chardef| defined token (there is no
% |\boxdef|).
%
% \subsection{Module utilities}
%
% \DescribeMacro{luatexbase.provides_module}
% |luatexbase.provides_module(|\meta{info}|)|\\
% This function is used by modules to identify themselves; the |info| should be
% a table containing information about the module. The required field
% |name| must contain the name of the module. It is recommended to provide a
% field |date| in the usual \LaTeX{} format |yyyy/mm/dd|. Optional fields
% |version| (a string) and |description| may be used if present. This
% information will be recorded in the log. Other fields are ignored.
%
% \noindent
% \DescribeMacro{luatexbase.module_info}
% \DescribeMacro{luatexbase.module_warning}
% \DescribeMacro{luatexbase.module_error}
% |luatexbase.module_info(|\meta{module}, \meta{text}|)|\\
% |luatexbase.module_warning(|\meta{module}, \meta{text}|)|\\
% |luatexbase.module_error(|\meta{module}, \meta{text}|)|\\
% These functions are similar to \LaTeX{}'s |\PackageError|, |\PackageWarning|
% and |\PackageInfo| in the way they format the output.  No automatic line
% breaking is done, you may still use |\n| as usual for that, and the name of
% the package will be prepended to each output line.
%
% Note that |luatexbase.module_error| raises an actual Lua error with |error()|,
% which currently means a call stack will be dumped. While this may not
% look pretty, at least it provides useful information for tracking the
% error down.
%
% \subsection{Callback management}
%
% \noindent
% \DescribeMacro{luatexbase.add_to_callback}
% |luatexbase.add_to_callback(|^^A
% \meta{callback}, \meta{function}, \meta{description}|)|
% Registers the \meta{function} into the \meta{callback} with a textual
% \meta{description} of the function. Functions are inserted into the callback
% in the order loaded.
%
% \noindent
% \DescribeMacro{luatexbase.remove_from_callback}
% |luatexbase.remove_from_callback(|\meta{callback}, \meta{description}|)|
% Removes the function with \meta{description} from the \meta{callback}.
% The removed function and its description 
% are returned as the results of this function.
%
% \noindent
% \DescribeMacro{luatexbase.in_callback}
% |luatexbase.in_callback(|\meta{callback}, \meta{description}|)|
% Checks if the \meta{description} matches one of the functions added
% to the list for the \meta{callback}, returning a boolean value.
%
% \noindent
% \DescribeMacro{luatexbase.disable_callback}
% |luatexbase.disable_callback(|\meta{callback}|)|
% Sets the \meta{callback} to \texttt{false} as described in the Lua\TeX{}
% manual for the underlying \texttt{callback.register} built-in. Callbacks
% will only be set to false (and thus be skipped entirely) if there are
% no functions registered using the callback.
%
% \noindent
% \DescribeMacro{luatexbase.callback_descriptions}
% A list of the descriptions of functions registered to the specified
% callback is returned. |{}| is returned if there are no functions registered.
%
% \noindent
% \DescribeMacro{luatexbase.create_callback}
% |luatexbase.create_callback(|\meta{name},meta{type},\meta{default}|)|
% Defines a user defined callback. The last argument is a default
% functtion of |false|.
%
% \noindent
% \DescribeMacro{luatexbase.call_callback}
% |luatexbase.create_callback(|\meta{name},\ldots|)|
% Calls a user defined callback with the supplied arguments.
%
% \endgroup
%
% \StopEventually{}
%
% \section{Implementation}
%
%
% \subsection{Plain \TeX\ support}
%
%    \begin{macrocode}
%<*plain>
%    \end{macrocode}
%
%    \begin{macrocode}
\edef\etatcatcode{\the\catcode`\@}
\catcode`\@=11
%    \end{macrocode}
%
%    \begin{macrocode}
\ifx\e@alloc\@undefined\else
  \expandafter\endinput
\fi
%    \end{macrocode}
%
% \subsubsection{Fixes to \texttt{etex.src}/\texttt{etex.sty}}
%
% These could and probably should be made directly in an
% update to etex.src which already has some Lua\TeX{}-specific
% code, but does not define the correct range for Lua\TeX{}.
%
%    \begin{macrocode}
% 2015-07-13 higher range in luatex
\edef \et@xmaxregs {\ifx\directlua\@undefined 32768\else 65536\fi}
% luatex/xetex also allow more math fam
\edef \et@xmaxfam {\ifx\Umathchar\@undefined\sixt@@n\else\@cclvi\fi}
%    \end{macrocode}
%
%    \begin{macrocode}
\count 270=\et@xmaxregs % locally allocates \count registers 32767, 32766, ...
\count 271=\et@xmaxregs % ditto for \dimen registers
\count 272=\et@xmaxregs % ditto for \skip registers
\count 273=\et@xmaxregs % ditto for \muskip registers
\count 274=\et@xmaxregs % ditto for \box registers
\count 275=\et@xmaxregs % ditto for \toks registers
\count 276=\et@xmaxregs % ditto for \marks classes
%    \end{macrocode}
%
%    \begin{macrocode}
% and 256 or 16 fam
\outer\def\newfam{\alloc@8\fam\chardef\et@xmaxfam}
%    \end{macrocode}
%
% End of proposed changes to \texttt{etex.src}
%
% \subsubsection{luatex specific settings}
% Switch to global cf |luatex.sty| to leave room for inserts
% not really needed for Lua\TeX{} but possibly most compatible
% with existing use.
%    \begin{macrocode}
\let\newcount\globcount
\let\newdimen\globdimen
\let\newskip\globskip
\let\newbox\globbox
%    \end{macrocode}
%
% Define|\e@alloc| as in latex (the existing macros in |etex.src|
% hard to extend to further register types as they assume specific
% 26x and 27x count range. For compatibility the existing register
% allocation is not changed.
%
%    \begin{macrocode}
\chardef\e@alloc@top=65535
\let\e@alloc@chardef\chardef
%    \end{macrocode}
%
%    \begin{macrocode}
\def\e@alloc#1#2#3#4#5#6{%
  \global\advance#3\@ne
  \e@ch@ck{#3}{#4}{#5}#1%
  \allocationnumber#3\relax
  \global#2#6\allocationnumber
  \wlog{\string#6=\string#1\the\allocationnumber}}%
%    \end{macrocode}
%
%    \begin{macrocode}
\gdef\e@ch@ck#1#2#3#4{%
  \ifnum#1<#2\else
    \ifnum#1=#2\relax
      #1\@cclvi
      \ifx\count#4\advance#1 10 \fi
    \fi
    \ifnum#1<#3\relax
    \else
      \errmessage{No room for a new \string#4}%
    \fi
  \fi}%
%    \end{macrocode}
%
% Two simple \LaTeX\ macros used in |ltlatex.sty|.
%    \begin{macrocode}
\long\def \@gobble #1{}
\long\def\@firstofone#1{#1}
%    \end{macrocode}
%
%    \begin{macrocode}
% Fix up allocations not to clash with |etex.src|.
%    \end{macrocode}
%
%    \begin{macrocode}
\newcount\e@alloc@attribute@count
\newcount\e@alloc@ccodetable@count
\newcount\e@alloc@luafunction@count
\newcount\e@alloc@whatsit@count
%    \end{macrocode}
%
%    \begin{macrocode}
% Input the main code.
\input ltluatex.sty\relax
%    \end{macrocode}
%
% Reset the catcode of |@|.
%    \begin{macrocode}
\catcode`\@=\etatcatcode\relax
%    \end{macrocode}
%
%    \begin{macrocode}
%</plain>
%    \end{macrocode}
%
% \subsection{luatexbase interface}
%
%    \begin{macrocode}
%<*emu>
\edef\emuatcatcode{\the\catcode`\@}
\catcode`\@=11
%    \end{macrocode}
%
%    \begin{macrocode}
% emu-luatexbase.sty v 0.01 2015/07/14
% minimal luatexbase emulation, for testing with
% current luaotfload package
%    \end{macrocode}
%
%
%    \begin{macrocode}
\ifx\e@alloc\@undefined
%    \end{macrocode}
%
%    \begin{macrocode}
% In pre 2014 latex, or plain TeX load |etex.{sty,src}|
%    \end{macrocode}
%
%    \begin{macrocode}
  \ifx\documentclass\@undefined
    \ifx\loccount\@undefined
      \input{etex.src}%
    \fi
    \catcode`\@=11 %
  \else
    \RequirePackage{etex}
  \fi
%    \end{macrocode}
% Load the code for the plain interface (which also works in old
% \LaTeX).
%    \begin{macrocode}
  \input{ltluatex.tex}%
\else
%    \end{macrocode}
% In \LaTeX\ releases from 2015/01/01 load |ltluatex|,
% load |ltluatex.sty| unless things already defined.
%    \begin{macrocode}
  \ifx\catcodetable@initex\@undefined
    \ifx\RequirePackage\@undefined
      \input{ltluatex.sty}%
    \else
      \RequirePackage{ltluatex}
    \fi
  \fi
\fi
%    \end{macrocode}
%
% Simple require wrapper so that |luaotfload| does get loaded.
%    \begin{macrocode}
\def\RequireLuaModule#1{\directlua{require("#1")}}
%    \end{macrocode}
%
% Extended catcode table support.  Use the names from the previous
% \textsf{luatexbase} and \textsf{luatex} packages.
%    \begin{macrocode}
\let\CatcodeTableIniTeX\catcodetable@initex
\let\CatcodeTableString\catcodetable@string
\let\CatcodeTableLaTeX\catcodetable@latex
\let\CatcodeTableLaTeXAtLetter\catcodetable@atletter
%    \end{macrocode}
%
% Additional tables declared in the previous interface.
%    \begin{macrocode}
\newcatcodetable\CatcodeTableOther
\setcatcodetable\CatcodeTableOther{%
  \catcodetable\CatcodeTableString
  \catcode32 12 }
%    \end{macrocode}
%
%    \begin{macrocode}
\newcatcodetable\CatcodeTableExpl
\setcatcodetable\CatcodeTableExpl{%
  \catcodetable\CatcodeTableLaTeX
  \catcode126 10 % tilde is a space char
  \catcode32  9  % space is ignored
  \catcode9   9  % tab also ignored
  \catcode95  11 % underscore letter
  \catcode58  11 % colon letter
}
%    \end{macrocode}
%
% Top level access to catcodetable stack.
%    \begin{macrocode}
\def\BeginCatcodeRegime#1{%
  \@pushcatcodetable
  \catcodetable#1\relax}
%    \end{macrocode}
%
%    \begin{macrocode}
\def\EndCatcodeRegime{%
  \@popcatcodetable}
%    \end{macrocode}
%
% The implementation of the stack is completely
% different, but usage should match.
%    \begin{macrocode}
\let\PushCatcodeTableNumStack\@pushcatcodetable
\let\PopCatcodeTableNumStack\@popcatcodetable
%    \end{macrocode}
%
% \subsubsection{Additional lua code}
%    \begin{macrocode}
\directlua{
%    \end{macrocode}
%
% Remove all registered callbacks, then disable.
% Set to false if optional second argument is |true|.
%    \begin{macrocode}
function luatexbase.reset_callback(name,make_false)
  for _,v in pairs(luatexbase.callback_descriptions(name))
  do
    luatexbase.remove_from_callback(name,v)
  end
  if make_false == true then
    luatexbase.disable_callback(name)
  end
end
%    \end{macrocode}
%
% Emulate the catcodetables table.
% Explicitly fill the table rather than rely on the metatable call to
% |registernumber| as that is unreliable on old Lua\TeX{}.
%    \begin{macrocode}
luatexbase.catcodetables=setmetatable(
 {['latex-package'] = \number\catcodetable@atletter,
  ini    = \number\CatcodeTableIniTeX,
  string = \number\CatcodeTableString,
  other  = \number\CatcodeTableOther,
  latex  = \number\catcodetable@latex,
  expl   = \number\CatcodeTableExpl,
  expl3  = \number\CatcodeTableExpl},
 { __index = function(t,key)
    return luatexbase.registernumber(key) or nil
  end}
)}
%    \end{macrocode}
%
% On old Lua\TeX{} workaround hashtable issues.
% Allocate in \TeX{}, and also directly add to |luatexbase.catcodetables|.
%    \begin{macrocode}
\ifnum\luatexversion<80 %
\def\newcatcodetable#1{%
  \e@alloc\catcodetable\chardef
    \e@alloc@ccodetable@count\m@ne{"8000}#1%
  \initcatcodetable\allocationnumber
  {\escapechar=\m@ne
  \directlua{luatexbase.catcodetables['\string#1']=%
    \the\allocationnumber}}%
}
\fi
%    \end{macrocode}
%
%    \begin{macrocode}
\directlua{
%    \end{macrocode}
%
% |priority_in_callback| returns position in the callback list.
% Not provided by default in |ltluatex| as usually it is just used 
% as a boolean test, for which |in_callback| is provided.
%    \begin{macrocode}
function luatexbase.priority_in_callback (name,description)
  for i,v in ipairs(luatexbase.callback_descriptions(name))
  do
    if v == description then
      return i
    end
  end
  return false
end
%    \end{macrocode}
% 
% The (unreleased) version 0.7 of \textsf{luatexbase} provided this
% boolean test under a different name, so we provide an alias here.
%    \begin{macrocode}
luatexbase.is_active_callback = luatexbase.in_callback
%    \end{macrocode}
%
% |ltluatex| implementation of |provides_module| does not return
% print functions so define modified version here.
%    \begin{macrocode}
luatexbase.base_provides_module=luatexbase.provides_module
function luatexbase.provides_module(info)
  luatexbase.base_provides_module(info)
  return
    function(s) return luatexbase.module_error(info.name, s) end,
    function(s) return luatexbase.module_warning(info.name, s) end,
    function(s) return luatexbase.module_info(info.name, s) end,
    function(s) return luatexbase.module_info(info.name, s) end
end
}
%    \end{macrocode}
%
% Same for attribute table as catcode tables. In old Lua\TeX{}, add to the
% |luatexbase.attributes| table directly.
%    \begin{macrocode}
\ifnum\luatexversion<80 %
\def\attribute#1{%
  \e@alloc\attribute\attributedef
    \e@alloc@attribute@count\m@ne\e@alloc@top#1%
  {\escapechar=\m@ne
  \directlua{luatexbase.attributes['\string#1']=%
    \the\allocationnumber}}%
}
\fi
%    \end{macrocode}
%
% Resolve name clashes and prefixed name issues.
%    \begin{macrocode}
\let\newluatexattribute\newattribute
\let\setluatexattribute\setattribute
\let\unsetluatexattribute\unsetattribute
\let\newluatexcatcodetable\newcatcodetable
\let\setluatexcatcodetable\setcatcodetable
\let\luatexbase@directlua\directlua
\let\luatebase@ensure@primitive\@gobble
%    \end{macrocode}
%
% reset catcode of |@|.
%    \begin{macrocode}
\catcode`\@=\emuatcatcode\relax
%    \end{macrocode}
%
%    \begin{macrocode}
%</emu>
%    \end{macrocode}
%
%    \begin{macrocode}
%<*emu-cmp,emu-mod,emu-loa,emu-reg,emu-att,emu-cct,emu-mcb>
%    \end{macrocode}
%
%    \begin{macrocode}
\ifx\RequirePackage\undefined
  \input{luatexbase.sty}%
\else
  \RequirePackage{luatexbase}
\fi
%    \end{macrocode}
%
%    \begin{macrocode}
%</emu-cmp,emu-mod,emu-loa,emu-reg,emu-att,emu-cct,emu-mcb>
%    \end{macrocode}
%
% \Finale
%
\else
  \RequirePackage{luatexbase}
\fi
%    \end{macrocode}
%
%    \begin{macrocode}
%</emu-cmp,emu-mod,emu-loa,emu-reg,emu-att,emu-cct,emu-mcb>
%    \end{macrocode}
%
% \Finale
%
\else
  \RequirePackage{luatexbase}
\fi
%    \end{macrocode}
%
%    \begin{macrocode}
%</emu-cmp,emu-mod,emu-loa,emu-reg,emu-att,emu-cct,emu-mcb>
%    \end{macrocode}
%
% \Finale
%
\else
  \RequirePackage{luatexbase}
\fi
%    \end{macrocode}
%
%    \begin{macrocode}
%</emu-cmp,emu-mod,emu-loa,emu-reg,emu-att,emu-cct,emu-mcb>
%    \end{macrocode}
%
% \Finale
